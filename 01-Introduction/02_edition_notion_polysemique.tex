\begin{frame}{L'édition: une notion polysémique}

Le terme édition - tout comme celui d’éditeur - est bien plus polysémique qu’on ne l’imagine au premier abord. Tour d’horizon de ce que l’on entend par éditeur, et les activités d’édition.

    
\end{frame}

\begin{frame}{Établir, arranger et transmettre les textes : l'éditeur savant
}



\begin{columns}[T,onlytextwidth]
\column{0.57\textwidth}
\begin{block}
\small
\textit{« Personne qui fait paraître un texte après l’avoir établi »} (le philologue, le chercheur, etc.).
\end{block}
\begin{block}{Pourquoi a-t-on besoin d'un tel éditeur?}
    \begin{itemize}
        \item Garantir la transmission d'un texte ;
        \item Les textes ne sont pas, par nature, fixes. Ex: \href{https://www.dhi.ac.uk/onlinefroissart/browsey.jsp?f=b&pb3=Chi-1_128r&pb2=B88-1_13v&GlobalWord=46794&div0=ms.f.transc.Bes-1&panes=4&GlobalMode=shf&img3=&disp2=pb&disp3=pb&div3=ms.f.transc.Chi-1&div2=ms.f.transc.B88-1&div1=ms.f.transc.Aus&img0=&disp0=pb&disp1=pb&pb1=Aus_1_153v&img2=&GlobalShf=1-315&pb0=Bes-1_156v&img1=}{Les chroniques de Froissart}
    \end{itemize}
\end{block}

\column{0.50\textwidth}
\includegraphics[width=.60\textwidth]{01-Introduction/Images/Porphyry.jpg}
\end{columns}
\end{frame}

\begin{frame}{L'éditeur dans la chaîne éditoriale}

\begin{itemize}
    \item Éditeur= "Personne ou société qui assure la publication et la mise en vente d’un ouvrage imprimé", rouage essentiel de l'énonciation éditoriale
\end{itemize}

\begin{center}
    \includegraphics[width=.90\textwidth]{01-Introduction/Images/chaineMathay.png}
\end{center}
    
\end{frame}

\begin{frame}{L'éditeur dans la chaîne éditoriale}
    \begin{itemize}
        \item Éditeur transforme le texte (manuscrit) en ouvrage (livre). Ex: les romans-feuilletons.
        \item Éditeur est là pour transformer l'écrivain en auteur.
    \end{itemize}


\textrightarrow{} Un des rôles clefs de l'éditeur et de l'édition = \textbf{légitimer}.

\end{frame}

