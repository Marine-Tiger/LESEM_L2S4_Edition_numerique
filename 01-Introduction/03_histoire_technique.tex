\begin{frame}{L'édition: une histoire technique}

\textrightarrow{} Un outil, un appareil technique pour produire et diffuser le texte.

\begin{block}{Outils d'édition possibles:}
\begin{itemize}
    \item Éditeur de texte ;
    \item Formateur de texte ;
    \item Traitement de texte.
\end{itemize}
\end{block}

\end{frame}

\begin{frame}{Éditeurs ou formateurs de texte}
\textrightarrow{} Outil qui met en forme un texte, à partir d'un fichier source, 
en texte brut mais contenant des indications de structures (balises).

\begin{center}
    \includegraphics[width=.50\textwidth]{01-Introduction/Images/latex.png}
    \includegraphics[width=.43\textwidth]{01-Introduction/Images/markdown.png}
\end{center}

\end{frame}

\begin{frame}{Traitement de texte}


\begin{itemize}
    \item Formatage du texte
    \item Affichage WYSIWYG (\textit{What You See Is What You Get}) du texte
    \item Impression finale
\end{itemize}
\begin{center}
    \includegraphics[width=.70\textwidth]{01-Introduction/Images/page_garde.png}
\end{center}


\textbf{\textrightarrow{} Imposent un format aux textes} 
\end{frame}
