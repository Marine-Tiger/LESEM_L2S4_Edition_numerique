\begin{frame}{Les trois fonctions éditoriales}
\begin{center}
    \textit{L'édition peut être comprise comme un \textbf{processus de médiation} qui permet à un contenu d'exister et d'être accessible. On peut distinguer trois étapes de ce processus qui correspondent à trois fonctions différentes de l'édition : une fonction de \textbf{choix et production}, une fonction de \textbf{légitimation} et une fonction de \textbf{diffusion}. Analyser ces trois fonctions nous permettra de comprendre à quoi sert l'édition pour s'interroger ensuite sur la façon dont les technologies numériques réagencent le processus en le transformant.}\\
    Epron et Vitali-Rosati, L’édition à l’époque du numérique, 2017
\end{center}
\end{frame}

\begin{frame}{Production des contenus}
\begin{block}{}
    \textbf{Éditer}\\
    \small
Choisir et produire. Renvoie à la définition historique de l'éditeur (aujourd'hui éditeur scientifique ou philologue) = établissement du texte en vue de sa publication.\\
     Il ne s'agit donc pas seulement d'un travail sur le fond du texte, mais également d'une conceptualisation de sa forme livresque.
\end{block}


\begin{block}{}
\textbf{Production}\\
\small
    Ensemble des dispositifs dans leurs aspects humains, institutionnels et techniques qui concourent à la création des contenus livresques.
\end{block}
\end{frame}

\begin{frame}{Production des contenus}

\begin{block}{}
\textbf{Qu'est-ce qui est digne d'être éditer?}\\
\begin{itemize}
    \item Ce qui est intéressant?
    \item Ce qui va donc trouver un public?
\end{itemize}
\end{block}

\begin{block}{}
    La fonction éditoriale doit assurer la bonne mise en forme des contenus en vue de leur publication.
\end{block}
    
\end{frame}

\begin{frame}{Diffusion des contenus}
\begin{center}
    La fonction de diffusion comporte \textbf{l'adresse}, \textbf{la distribution} et tous les dispositifs qui tendent à rendre un contenu \textbf{matériellement accessible et visible}. Cette fonction comprend de fait l'ensemble des actions de \textbf{publicisation} du contenu. Il est important, pour un éditeur littéraire en particulier, d'apprendre à opérer la distinction entre le lecteur idéal supposé par toute oeuvre littéraire, une figure idéale / idéalisée notamment par l'auteur, de la communauté des lecteurs réels, avec leurs attentes et leurs usages culturels.
\end{center}
    
\end{frame}

\begin{frame}{Validation des contenus}

\begin{center}
    La fonction de légitimation renvoie aux dispositifs d'autorité qui permettent au public de \textbf{se repérer} dans les contenus, de \textbf{choisir les textes} et \textbf{d'évaluer leur fiabilité}. Si elle est supposée laisser des indices clairs sur leur valeur et finalement sur leur sens, on voit que la fonction de légitimation s'appuie sur un imaginaire collectif et individuel très fort.
\end{center}
    
\end{frame}