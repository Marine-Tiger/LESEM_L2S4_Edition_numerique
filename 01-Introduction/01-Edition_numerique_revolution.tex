\begin{frame}{Une histoire des remédiations du livre}

\begin{center}
    \textit{"On présente le texte électronique comme une révolution. L'histoire du livre en a vu d'autres !"}\\
    Roger Chartier, Le livre en révolutions.
\end{center}

\end{frame}

\begin{frame}{Une histoire des remédiations du livre}


    \begin{itemize}
        \item Histoire matérielle et technique
\item Histoire culturelle de l'écriture et de la lecture
\item Histoire de la littérature
    \end{itemize}


\end{frame}

\begin{frame}{1ère époque: du \textit{volumen} au codex}


\begin{columns}[T,onlytextwidth]
\column{0.45\textwidth}
\includegraphics[width=.45\textwidth]{01-Introduction/Images/volumen2.jpeg}
\begin{block}{Volumen}
\small
\begin{itemize}
\item  rouleaux de plusieurs mètres que l'on déroule et renroule au fur et à mesure de manière horizontale (vers la droite)
\end{itemize}
\end{block}

\column{0.45\textwidth}
\includegraphics[width=.86\textwidth]{01-Introduction/Images/rouleau-codex.jpg}
\begin{block}{Codex}
\small
\begin{itemize}
\item mot latin, désigne le livre formé de feuilles pliées et assemblées en cahiers, et couvert d'une reliure tel que nous le connaissons
\end{itemize}
\end{block}
\end{columns}

\end{frame}


\begin{frame}{1ère époque: du \textit{volumen} au codex}

\begin{block}{Codex: Révolution technologique et intellectuelle}
    \begin{itemize}
        \item Invention de la page (et de l'architecture du savoir: sommaires, index, pages de titre et couverture)
        \item De nouveaux gestes de lecture et d'annotation
        \item La trace du volumen dans le lexique éditorial ("le volume")
    \end{itemize}
\end{block}

\end{frame}

\begin{frame}{2ème époque: du manuscrit à l'imprimé}

\begin{block}{Grâce à l'automatisation}
    \begin{itemize}
        \item Vers une stabilisation des textes
        \item Vers une uniformisation de l'édition
        \item Naissance d'une industrie culturelle
        \item Un changement d'échelle de la diffusion des savoirs
    \end{itemize}
\end{block}

\end{frame}

\begin{frame}{2ème époque: du manuscrit à l'imprimé}
\textbf{Les Incunables}\\
Les premiers livres fabriqués dans les décennies qui ont suivi l'invention de Gutenberg.
\begin{center}
    \includegraphics[width=.80\textwidth]{01-Introduction/Images/ex_incunable.png}
\end{center}    
\end{frame}

\begin{frame}{3ème époque: de l'imprimé au numérique (et vice-versa...)}
\begin{center}
    \includegraphics[width=.60\textwidth]{01-Introduction/Images/livrenum.jpeg}
\end{center}

\end{frame}

\begin{frame}{3ème époque: de l'imprimé au numérique (et vice-versa...)}
\begin{block}{Le "livre numérique" n'existe pas...}
    \begin{itemize}
        \item Une "adaptation" de l'imprimé à un mode de consultation/diffusion numérique (= livres homothétiques)
        \item Des propositions techniques et conceptuelles natives numériques (oeuvres hypermédiatiques, livres-appli, etc.)
        \item Une production hybride : complémentarité imprimé & numérique ; publication assistée par ordinateur (PAO); \textit{Web to print}
    \end{itemize}
\end{block}
\end{frame}

\begin{frame}{3ème époque: de l'imprimé au numérique (et vice-versa...)}
\textbf{De l'imprimé au numérique: ruptures et continuités}\\


\textit{L’inscription du texte sur l’écran crée une distribution, une organisation, une structuration du texte qui n’est pas du tout la même que celle que rencontrait le lecteur dans le rouleau de l’Antiquité ou le lecteur médiéval, moderne et contemporain dans le livre manuscrit ou imprimé, où le texte est organisé à partir de sa structure en cahiers, feuillets et pages.
la révolution du texte électronique est une révolution des structures du support matériel de l’écrit comme des manières de lire.}

Roger Chartier, Le livre en révolutions

\end{frame}

\begin{frame}{Le tournant numérique de la fabrique éditoriale : la PAO}
\textrightarrow{} Préparation des documents destinés à l'impression à l'aide d'un ordinateur en lieu et place des procédés historiques de la typographique et de la photocomposition.\\


\textbf{Ex:} Adobe In Design, Microsft Publisher, Word ...

\begin{center}
    \includegraphics[width=.35\textwidth]{01-Introduction/Images/prepaCopie_PAO.png}
    \includegraphics[width=.45\textwidth]{01-Introduction/Images/CouvIndesign-CreerCouv-1.jpg}
\end{center}
\end{frame}

