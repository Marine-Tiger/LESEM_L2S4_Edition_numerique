\begin{frame}{Des oeuvres éphémères?}

\begin{columns}[T,onlytextwidth]
\column{0.50\linewidth}
\centering
\includegraphics[width=.80\textwidth]{03_Axe1_formats/Images/bayard_livres_non_lus.png}

\column{0.50\linewidth}
\begin{itemize}
\item Pierre Bayard, 2007
    \item Essai qui prône la "non-lecture"
\end{itemize}
\end{columns}
\textbf{Raisons multiples} : oeuvres débranchées du web, oeuvres censurées. Mais très souvent des raisons d'obsolescence, liées au manque de pérennité des formats.
\end{frame}

\begin{frame}{Des oeuvres éphémères?}
    \begin{block}{Exemple: la fin de d'Adobe Flash Player}
        \begin{itemize}
            \item \textit{plug-in} permettant d'afficher/lire du contenu multimédia
            \item Début 2021: plus supporté par les navigateurs web
            \item Impossible de lire des images et des vidéos.
        \end{itemize}
    \end{block}
\textrightarrow{} En quelques semaines, la littérature numérique créée à l'aide de ce logiciel n'était plus accessible au public (Exemple \textit{Electronic Literature Organization}).
\end{frame}

\begin{frame}{Des oeuvres éphémères?}
\textbf{Création de WARC}
\begin{columns}[T,onlytextwidth]
\column{0.50\linewidth}
\begin{itemize}
    \item Publié à partir de 2009
    \item Développé dans le cadre des collectes de données du web
    \item Conçu pour la préservation du web
    \item A une notice internationale
\end{itemize}

\column{0.50\linewidth}
\centering
\includegraphics[width=.90\textwidth]{03_Axe1_formats/Images/WARC.png}
\end{columns}
\end{frame}

\begin{frame}{Les archives du web de la BNF}
    \textbf{Exemple de la collecte des Skyblogs}\\
    Emmanuelle Bermès, à la rencontre des premiers skyblogs archivés, 2025, \url{https://webcorpora.hypotheses.org/7348}\\
    \vspace{5mm}
    \centering
    \includegraphics[width=.80\textwidth]{03_Axe1_formats/Images/ex_skyblog.jpg}
\end{frame}

% Des oeuvres dont il nous reste des fragments (archives personnelles) Archives institutionnelles : BNF Format internationnal des archives = WARC. On est pas dans le paradigme de l'archive traditionnelle, qui est un prélèvement, mais dans une captation et représentation dans un autre format (un peu comme une photographie si l'on veut). Archives incomplètes, parfois fautives
% Des oeuvres que l'on essaie de restaurer, mais plus véritablement de re-créer, reconstruire dans d'autres formats. C'est ce qu'on appelle les émulateurs.

% De ce point de vue, un trauma dans la communauté elit = Flash

% Le 31 décembre 2020, Adobe a cessé de prendre en charge le logiciel Flash, une plateforme de premier plan pour l'art numérique très populaire à la fin du XXe siècle et au début du XXIe siècle. En quelques semaines, la littérature numérique créée à l'aide de ce logiciel n'était plus accessible au public, y compris les 447 œuvres que l'Electronic Literature Organization (ELO) avait rassemblées dans ses archives. À la fin du mois de janvier 2021, l'Electronic Literature Lab a commencé à travailler sérieusement à la restauration des archives Flash de l'ELO à l'aide de diverses méthodes : Ruffle.rs, Conifer, Webrecorder et des enregistrements vidéo réalisés avec le navigateur Pale Moon et la Wayback Machine.

% Réflexion sur notre dépendance à des formats qui ne sont pas standards, et qui appartiennent à des entreprises.