\begin{frame}{Que signifie choisir un format?}

    \textbf{Choisir des formats = choisir des outils éditoriaux} avec lesquels nous allons travailler, mais également diffuser une oeuvre, voire la légitimer. \\ 
    \textrightarrow{} Les formats ont fait parfois l'objet d'une invisibilisation dont les conséquences pratiques, mais également politiques, sont à souligner.\\

\vspace{2mm}
\begin{block}{Exemple: hyperfiction \textit{Conduit d'aération}, le "récit pour Ipad"}
\vspace{3mm}
\centering
\includegraphics[width=0.55\textwidth]{03_Axe1_formats/Images/recit_ipad.png}
\url{https://vimeo.com/74614087?fl=pl&fe=ti}
\end{block}
\end{frame}

\begin{frame}{Importance des technologies utilisées}
\begin{block}{}
    \centering
    {\small
    \textit{[A]u-delà de la compétence technique de l’auteur, la valeur littéraire relèverait en particulier des connotations attribuées au prestige d’une technologie ou d’une marque à laquelle il associe son nom. Dans ce cadre, l’œuvre littéraire apparaîtrait comme indissociable de la strate des discours tenus à son sujet comme production technique.}}\\
    \textbf{Étienne Candel et Gustavo Gomez-Mejia, « \textit{Écrire l’auteur : la pratique éditoriale comme construction socioculturelle de la littérarité des textes sur le Web} », L’auteur en réseau, les réseaux de l’auteur, 2013.}
\end{block}
\begin{block}{}
     \textrightarrow{} La technique: pas que le code et la programmation, comprend aussi une part discursive et culturelle majeure. \\
     \textrightarrow{} Rappeler le capital symbolique que certaines maisons d’édition exercent.
\end{block}
\end{frame}

\begin{frame}{Être éditeur à l'ère numérique: comprendre les enjeux des formats}

\textrightarrow{} Le format fait partie de \textbf{la construction sémiotique} de plusieurs œuvres de littérature numérique. \\
\vspace{3mm}
\textrightarrow{} \textbf{Enjeux politiques et culturels}: GAFAM mettent en péril une bibliodiversité numérique encore émergente (monopole, invisibilisation)\\
\vspace{3mm}
\textrightarrow{}\textbf{Efficacité de ces outils}: gains de temps dans la production et diffusion des contenus. \\
\vspace{3mm}
\textrightarrow{} Tendre vers \textbf{une utilisation raisonnée} de ces outils?
\end{frame}

\begin{frame}{Les formats : une guerre d'édition à l'époque numérique}


\begin{table}
    \begin{tabular}{p{5cm}|p{6cm}}
       \textbf{GAFAM}  & \textbf{Communautés (ex: W3C)} \\
       \hline
        Logique commerciale& 
Logique éditoriale\\
Format propriétaire (payant) & Format ouvert (gratuit)\\
Facile d'accès pour les consommateurs & Développement de standards
    \end{tabular}
\end{table}

\textrightarrow{} Réflexion sur la propriété intellectuelle des outils.

    
\end{frame}

\begin{frame}{Aujourd'hui, une page web est composée de...}
    \begin{block}{}
        \begin{itemize}
            \item HTML (HyperText Markup Language)
            \item CSS (Cascading Style Sheets)
            \item Javascript
        \end{itemize}
    \end{block}
    \begin{block}{Il s'agit de formats standards}
        \begin{itemize}
            \item \textbf{HTML} = code utilisé pour structurer une page web et son contenu ;
            \item \textbf{CSS} = Appliquer des styles à des éléments de page: ce langage décrit la présentation des documents HTML
            \item \textbf{Javascript} = interactivité
        \end{itemize}
    \end{block}
\end{frame}

\begin{frame}{Le W3C : Autorité de régulation des standards informatiques}

\begin{columns}[T,onlytextwidth]
\column{0.50\linewidth}
\small
\begin{itemize}
\item World Wide Web Consortium
\item Tim Berners-Lee
\item Définit les standards du web
\item Procédures de gouvernance ouverte et prend ses décisions par consensus
\end{itemize}

\column{0.50\linewidth}
\includegraphics[width=0.80\textwidth]{03_Axe1_formats/Images/W3C.png}
\end{columns}
\vspace{5mm}
\textrightarrow{} \textit{(...) soumise à des enjeux économiques forts, notamment parce que les grands acteurs du web (Google, Facebook, Microsoft, etc.) peuvent imposer des standards de fait, sans passer par les procédures de normalisation internationale du W3C. (Dominique Cardon, Culture numérique.)}
    
\end{frame}

\begin{frame}{Les guerres de standards : le cas du livre numérique}

\begin{itemize}
    \item .epub : format standard, ouvert
    \item .azw : format lancé par Amazon, format propriétaire
\end{itemize}
\end{frame}

\begin{frame}{Apprendre à redévelopper une pensée critique de nos outils}
\begin{block}{}
    \centering
{\small \textit{La prolétarisation est, d’une manière générale, ce qui consiste à priver un sujet (producteur, consommateur, concepteur) de ses savoirs (savoir-faire, savoir-vivre, savoir concevoir et théoriser).}}\\
\textbf{Bernard Stiegler}
\end{block}
\begin{block}{}
\begin{itemize}
    \item Distinguer prolétarisation // paupérisation.
    \item Préférence pour \textbf{la maîtrise d'un outil} qui va dominer le marché plutôt que d'\textbf{enseigner les compétences} visées par ces outils (correction, édition, mise en forme etc...)\\
    \textrightarrow{} Éditeurs se trouvent "\textit{\textbf{prolétarisés}}"
\end{itemize}
\end{block}
\end{frame}

\begin{frame}{Expérience : révéler la composition d'un fichier Word}
\begin{itemize}
        \item Prendre un fichier Word (.docx) 
        \item Changer l'extension en.zip
        \item Ouvrir le fichier zip
    \end{itemize}
\begin{columns}[T,onlytextwidth]
\column{0.50\linewidth}
\centering
\includegraphics[width=5cm]{03_Axe1_formats/Images/experience1_bis.png}

\column{0.50\linewidth}
\centering
\includegraphics[width=6cm]{03_Axe1_formats/Images/experience1.png}
\end{columns}
    
\end{frame}

\begin{frame}{Expérience : révéler la composition d'un fichier Word}
\centering
\includegraphics[width=12cm]{03_Axe1_formats/Images/experience2.png}
\end{frame}

\begin{frame}{Expérience : révéler la composition d'un fichier Word}
\begin{block}{}
    \centering
\includegraphics[width=6cm]{03_Axe1_formats/Images/experience2_bis.png}
\includegraphics[width=6cm]{03_Axe1_formats/Images/experience3.png}
\end{block}
\vspace{5mm}
\textrightarrow{} Documents souvent mal structurés, difficilement interprétables par la machine sur le long terme, donc non interopérables et non pérennes.\\
\textrightarrow{} Éloignement progressif entre nos actions sur l'interface et la machine.

\end{frame}
\begin{frame}{Expérience : révéler la composition d'un fichier Word}

\begin{block}{Word:}
    \begin{itemize}
        \item Au début, développé avec son propre langage
        \item Création du .docx basé sur le langage XML
    \end{itemize}
\end{block}
\begin{block}{Formats ouverts:}
\begin{itemize}
     \item Fichiers encodés de façon transparente
    \item Code est dans le domaine public
    \item Interropérables
    \item A privilégier pour la préservation et le partage des données 
\end{itemize}
   
\end{block}

\begin{block}{En conclusion:}
    \begin{itemize}
        \item Docx = format propriétaire, payant, mais reste en théorie ouvert.
    \end{itemize}
\end{block}

\end{frame}