\begin{frame}{Exemple: mauvais encodage}
    \begin{block}{}
    \centering
    {\small
    \textit{Les contenus numériques sont, par nature, encodés. Pour pouvoir être partagée, une information doit être structurée selon des standards : les formats. Le choix d’un format a des implications profondes : les informations que l’on peut transmettre changent, ainsi que leur lisibilité, leur universalité, leur agencement, leur transportabilité, leur transformabilité, etc.}\\
    }
\end{block}
\begin{block}{}
    \textrightarrow{} Ouverture d'un fichier texte avec le mauvais encodage.\\
    \textbf{Résultat =} Gros problème de lisibilité, surtout les accents.
\end{block}
\end{frame}
\begin{frame}{Exemple: mauvais encodage}
    \begin{block}{}
    \centering
    {\small
    \textit{Les contenus numériques sont, par nature, encodés. Pour pouvoir être partagée, une information doit être structurée selon des standards : les formats. Le choix d’un format a des implications profondes : les informations que l’on peut transmettre changent, ainsi que leur lisibilité, leur universalité, leur agencement, leur transportabilité, leur transformabilité, etc.}\\
    }
    \textbf{Viviane Boulétreau et Benoît Habert, \textit{"Les formats", Qu'est-ce que l'édition numérique?}, PUM 2014.}
\end{block}
\begin{block}{}
    \textrightarrow{} Ouverture avec le bon encodage (Unicode UTF-8, qui permet entre autre de bien encoder les accents)
\end{block}
\end{frame}

\begin{frame}{Catégories de formats}
    \begin{block}{}
        \textbf{Format =} Extension à la fin du fichier.\\
    \end{block}
    \begin{block}{}
        Distingue plusieurs catégories de format:
        \begin{itemize}
            \item Les \textbf{« formats propriétaires »} : défini par une organisation ou un propriétaire privé qui détient les droits de propriété intellectuelle. Exemple: AZW (KINDLE) / DOCX (word) / PDF 
            \item Les \textbf{« formats libres ou ouverts »} : Publié et libre de droit, sans restriction d'usage. Exemple: ODT, EPUB \\
        \end{itemize}
    \end{block}
    \textrightarrow{} Droit d'auteurs avec les formats: impossible d'ouvrir certains fichiers sans posséder le logiciel. Exemple: impossible d'ouvrir un fichier .pages avec windows.
\end{frame}

\begin{frame}{Enjeux éditoriaux liés aux choix de formats}
    \begin{itemize}
        \item \textbf{Esthétique} = la "forme" livresque. Certains formats numériques permettent de réaliser plus ou moins de formes éditoriales. \\
        Problèmes: longévité, manipulabilité, intérropérabilité?.
        \item \textbf{Manipulabilité} = capacité à travailler le texte / sur le texte : surligner, annoter, marquer des pages, citer les pages...\\
        \textrightarrow{} Capacité à lire le texte tout court. 
        \item \textbf{Interopérabilité} = capacité à fonctionner sur plusieurs machines et plusieurs environnement.
        \item \textbf{Ouverture (fermeture) du format} = rejoint l'interopérabilité : formats et polices libres.
        \item \textbf{Pérennité} = assurer la longévité d'une oeuvre.
    \end{itemize}
\end{frame}