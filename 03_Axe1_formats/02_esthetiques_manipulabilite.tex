\begin{frame}{Enjeux esthétiques et de manipulabilité : la \textit{forme} livre en question}
    \centering
    La remédiation numérique du livre pose des questions conceptuelles et formelles passionnantes sur nos attentes vis à vis du média livresque : à quoi doit ressembler un livre ? Quels usages en faisons-nous ? En quoi le paradigme de l'imprimé détermine-t-il notre imaginaire et notre usage du livre ?
\end{frame}

\begin{frame}{Enjeux esthétiques et de manipulabilité : la \textit{forme} livre en question}
    \begin{block}{}
    \textbf{Atelier: Lectures croisées des "versions" d'un texte en différents formats, à partir de \textit{Mémoire vive} de Pierre Ménard.}
    
    \end{block}
    \centering
        \includegraphics[width=0.50\textwidth]{03_Axe1_formats/Images/abrupt.png}
\end{frame}
\begin{frame}{Enjeux esthétiques et de manipulabilité : la \textit{forme} livre en question}
    \textbf{Conclusion}\\
    Les \textit{formats} éditoriaux permettent de jouer avec les formes esthétiques du livre, en modifiant l'énonciation éditoriale et les conditions de réception des textes. Les formats, de ce point de vue, déterminent les conditions de réception des textes ainsi qu'un pacte de lecture. Pour l'auteur et l'éditeur, il s'agit d'opérer des choix de diffusion (adresse, appropriabilité du texte).
\end{frame}
