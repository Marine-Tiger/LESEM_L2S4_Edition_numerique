\documentclass[10pt]{beamer}

\documentclass[svgnames,smaller]{beamer}

\usetheme{metropolis}

\usepackage[utf8]{inputenc}
\usepackage[russian,french]{babel}

\usepackage[T1]{fontenc}
\usepackage{geometry}
\geometry{
    left=2.5mm,
    right=2.5mm,
    top=0mm,
    bottom=0mm
}
\usepackage{amssymb} % pour \blackdiamond
\usepackage{tikz}
\usepackage{graphicx,animate}

\newcommand{\replacementchar}{%
  \tikz[baseline=-0.6ex]{
    %\node[inner sep=0pt] (d) {\Large$\blackdiamond$};
    \node[inner sep=0pt] (d) {\Large$\blacksquare$};
    \node[white, font=\scriptsize\bfseries] at (d.center) {?};
  }%
}
\usepackage{tabularx}
%\usepackage[style=alphabetic,citestyle=authoryear]{biblatex}
%\addbibresource{bibliographie.bib}

\usepackage{xspace}
\newcommand{\themename}{\textbf{\textsc{metropolis}}\xspace}
\title{L2S4: Édition numérique\\
TD 3: Choisir son format\\
\small
Comprendre les enjeux éditoriaux des formats numériques
}

\date{}
\author{Marine Tiger\\  \quad {marine.tiger@sorbonne-universite.fr\\}}

\institute{Littératures françaises et comparée ED19, CELLF, Sorbonne Université}

\begin{document}
\maketitle
\setbeamertemplate{section in toc}[sections numbered]

\section{Introduction}
\begin{frame}{Rappel: expérience de Goldsmith}
     \centering
    \includegraphics[width=.80\textwidth]{02_Axe1_histoire_outils_informatiques/Images/clair_obscur.jpg}
\end{frame}

\begin{frame}{Rappel: expérience de Goldsmith}
     \centering
    \includegraphics[width=12cm]{02_Axe1_histoire_outils_informatiques/Images/code_img.png}
    
    \includegraphics[width=12cm]{02_Axe1_histoire_outils_informatiques/Images/code_img_modif.png}
\end{frame}

\begin{frame}{Rappel: expérience de Goldsmith}
     \centering
    \includegraphics[width=.80\textwidth]{02_Axe1_histoire_outils_informatiques/Images/after_modif.png} 
\end{frame}

\begin{frame}{Quelques enseignements}
    \begin{itemize}
        \item Tout les contenus numériques sont, par nature, \textbf{encodés} ;
        \item Pour pouvoir être lus, partagés, ces contenus sont structurés selon des \textbf{standards} : les formats ;
        \item Invisibles à l'usager.
    \end{itemize}
\end{frame}

\begin{frame}{Exemples de formats}
\begin{block}{}
    \centering
    .pdf, .epub, .docx, .odt, .page, .azw etc...   
\end{block}
 \vspace{5mm} %5mm vertical space
\textbf{ \textrightarrow{} Quelles possibilités d'action peuvent-elles avoir sur les contenus?}


\end{frame}

\begin{frame}{Possibilités d'action sur les contenus}
    \begin{itemize}
        \item Certains sont \textbf{éditables} (docx, odt, doc, etc.) \textrightarrow{} permet de les \textbf{manipuler} (écrire, récrire, etc.);
        \item Certains garantissent une "\textbf{stabilisation}" du texte (PDF) \textrightarrow{} donne un statut plus proche du "livre" (enjeux de clôture et de stabilisation);
        \item Certains sont étroitement \textbf{associés à des environnements techniques} \textrightarrow{} un fichier pages ne peut être ouvert que sur un Mac.
    \end{itemize}
\end{frame}

\begin{frame}{Définitions de format}
    \begin{block}{}
    \centering
    {\small
    \textit{Dérivée de forme, la notion de format (formato) désigne à l’origine les « dimensions du papier », puis en est venue à prendre le sens de « mesure » et de « dimension ». Le format permet de cadrer et d’écarter des formes potentielles. Le format ne donc doit pas être seulement envisagé comme simple véhicule, ensemble de dimensions ou normes techniques: il s’agit de le considérer comme matrice, médiateur, cadre opératoire (Brulé & Masure 2015), ou « condition de possibilité » (Huyghe 2015).}}\\
        \textbf{Anthony Masure et Alexandre Saint-Jevin, "\textit{Formes, formats, formatage : vers un design des sciences}", \textit{Les devenirs numériques des patrimoines}, 2022.}
    \end{block}
\end{frame}

\section{L'édition numérique : une question de forme et de formats}
\begin{frame}{Exemple: mauvais encodage}
    \begin{block}{}
    \centering
    {\small
    \textit{Les contenus numériques sont, par nature, encodés. Pour pouvoir être partagée, une information doit être structurée selon des standards : les formats. Le choix d’un format a des implications profondes : les informations que l’on peut transmettre changent, ainsi que leur lisibilité, leur universalité, leur agencement, leur transportabilité, leur transformabilité, etc.}\\
    }
\end{block}
\begin{block}{}
    \textrightarrow{} Ouverture d'un fichier texte avec le mauvais encodage.\\
    \textbf{Résultat =} Gros problème de lisibilité, surtout les accents.
\end{block}
\end{frame}
\begin{frame}{Exemple: mauvais encodage}
    \begin{block}{}
    \centering
    {\small
    \textit{Les contenus numériques sont, par nature, encodés. Pour pouvoir être partagée, une information doit être structurée selon des standards : les formats. Le choix d’un format a des implications profondes : les informations que l’on peut transmettre changent, ainsi que leur lisibilité, leur universalité, leur agencement, leur transportabilité, leur transformabilité, etc.}\\
    }
    \textbf{Viviane Boulétreau et Benoît Habert, \textit{"Les formats", Qu'est-ce que l'édition numérique?}, PUM 2014.}
\end{block}
\begin{block}{}
    \textrightarrow{} Ouverture avec le bon encodage (Unicode UTF-8, qui permet entre autre de bien encoder les accents)
\end{block}
\end{frame}

\begin{frame}{Catégories de formats}
    \begin{block}{}
        \textbf{Format =} Extension à la fin du fichier.\\
    \end{block}
    \begin{block}{}
        Distingue plusieurs catégories de format:
        \begin{itemize}
            \item Les \textbf{« formats propriétaires »} : défini par une organisation ou un propriétaire privé qui détient les droits de propriété intellectuelle. Exemple: AZW (KINDLE) / DOCX (word) / PDF 
            \item Les \textbf{« formats libres ou ouverts »} : Publié et libre de droit, sans restriction d'usage. Exemple: ODT, EPUB \\
        \end{itemize}
    \end{block}
    \textrightarrow{} Droit d'auteurs avec les formats: impossible d'ouvrir certains fichiers sans posséder le logiciel. Exemple: impossible d'ouvrir un fichier .pages avec windows.
\end{frame}

\begin{frame}{Enjeux éditoriaux liés aux choix de formats}
    \begin{itemize}
        \item \textbf{Esthétique} = la "forme" livresque. Certains formats numériques permettent de réaliser plus ou moins de formes éditoriales. \\
        Problèmes: longévité, manipulabilité, intérropérabilité?.
        \item \textbf{Manipulabilité} = capacité à travailler le texte / sur le texte : surligner, annoter, marquer des pages, citer les pages...\\
        \textrightarrow{} Capacité à lire le texte tout court. 
        \item \textbf{Interopérabilité} = capacité à fonctionner sur plusieurs machines et plusieurs environnement.
        \item \textbf{Ouverture (fermeture) du format} = rejoint l'interopérabilité : formats et polices libres.
        \item \textbf{Pérennité} = assurer la longévité d'une oeuvre.
    \end{itemize}
\end{frame}

\section{Enjeux esthétiques et de manipulabilité : la forme livre en
question}
\begin{frame}{Enjeux esthétiques et de manipulabilité : la \textit{forme} livre en question}
    \centering
    La remédiation numérique du livre pose des questions conceptuelles et formelles passionnantes sur nos attentes vis à vis du média livresque : à quoi doit ressembler un livre ? Quels usages en faisons-nous ? En quoi le paradigme de l'imprimé détermine-t-il notre imaginaire et notre usage du livre ?
\end{frame}

\begin{frame}{Enjeux esthétiques et de manipulabilité : la \textit{forme} livre en question}
    \begin{block}{}
    \textbf{Atelier: Lectures croisées des "versions" d'un texte en différents formats, à partir de \textit{Mémoire vive} de Pierre Ménard.}
    
    \end{block}
    \centering
        \includegraphics[width=0.50\textwidth]{03_Axe1_formats/Images/abrupt.png}
\end{frame}
\begin{frame}{Enjeux esthétiques et de manipulabilité : la \textit{forme} livre en question}
    \textbf{Conclusion}\\
    Les \textit{formats} éditoriaux permettent de jouer avec les formes esthétiques du livre, en modifiant l'énonciation éditoriale et les conditions de réception des textes. Les formats, de ce point de vue, déterminent les conditions de réception des textes ainsi qu'un pacte de lecture. Pour l'auteur et l'éditeur, il s'agit d'opérer des choix de diffusion (adresse, appropriabilité du texte).
\end{frame}


\section{Enjeux d’intéropérabilité et d’ouverture : le formatage de l’édition en question}
\begin{frame}{Que signifie choisir un format?}

    \textbf{Choisir des formats = choisir des outils éditoriaux} avec lesquels nous allons travailler, mais également diffuser une oeuvre, voire la légitimer. \\ 
    \textrightarrow{} Les formats ont fait parfois l'objet d'une invisibilisation dont les conséquences pratiques, mais également politiques, sont à souligner.\\

\vspace{2mm}
\begin{block}{Exemple: hyperfiction \textit{Conduit d'aération}, le "récit pour Ipad"}
\vspace{3mm}
\centering
\includegraphics[width=0.55\textwidth]{03_Axe1_formats/Images/recit_ipad.png}
\url{https://vimeo.com/74614087?fl=pl&fe=ti}
\end{block}
\end{frame}

\begin{frame}{Importance des technologies utilisées}
\begin{block}{}
    \centering
    {\small
    \textit{[A]u-delà de la compétence technique de l’auteur, la valeur littéraire relèverait en particulier des connotations attribuées au prestige d’une technologie ou d’une marque à laquelle il associe son nom. Dans ce cadre, l’œuvre littéraire apparaîtrait comme indissociable de la strate des discours tenus à son sujet comme production technique.}}\\
    \textbf{Étienne Candel et Gustavo Gomez-Mejia, « \textit{Écrire l’auteur : la pratique éditoriale comme construction socioculturelle de la littérarité des textes sur le Web} », L’auteur en réseau, les réseaux de l’auteur, 2013.}
\end{block}
\begin{block}{}
     \textrightarrow{} La technique: pas que le code et la programmation, comprend aussi une part discursive et culturelle majeure. \\
     \textrightarrow{} Rappeler le capital symbolique que certaines maisons d’édition exercent.
\end{block}
\end{frame}

\begin{frame}{Être éditeur à l'ère numérique: comprendre les enjeux des formats}

\textrightarrow{} Le format fait partie de \textbf{la construction sémiotique} de plusieurs œuvres de littérature numérique. \\
\vspace{3mm}
\textrightarrow{} \textbf{Enjeux politiques et culturels}: GAFAM mettent en péril une bibliodiversité numérique encore émergente (monopole, invisibilisation)\\
\vspace{3mm}
\textrightarrow{}\textbf{Efficacité de ces outils}: gains de temps dans la production et diffusion des contenus. \\
\vspace{3mm}
\textrightarrow{} Tendre vers \textbf{une utilisation raisonnée} de ces outils?
\end{frame}

\begin{frame}{Les formats : une guerre d'édition à l'époque numérique}


\begin{table}
    \begin{tabular}{p{5cm}|p{6cm}}
       \textbf{GAFAM}  & \textbf{Communautés (ex: W3C)} \\
       \hline
        Logique commerciale& 
Logique éditoriale\\
Format propriétaire (payant) & Format ouvert (gratuit)\\
Facile d'accès pour les consommateurs & Développement de standards
    \end{tabular}
\end{table}

\textrightarrow{} Réflexion sur la propriété intellectuelle des outils.

    
\end{frame}

\begin{frame}{Aujourd'hui, une page web est composée de...}
    \begin{block}{}
        \begin{itemize}
            \item HTML (HyperText Markup Language)
            \item CSS (Cascading Style Sheets)
            \item Javascript
        \end{itemize}
    \end{block}
    \begin{block}{Il s'agit de formats standards}
        \begin{itemize}
            \item \textbf{HTML} = code utilisé pour structurer une page web et son contenu ;
            \item \textbf{CSS} = Appliquer des styles à des éléments de page: ce langage décrit la présentation des documents HTML
            \item \textbf{Javascript} = interactivité
        \end{itemize}
    \end{block}
\end{frame}

\begin{frame}{Le W3C : Autorité de régulation des standards informatiques}

\begin{columns}[T,onlytextwidth]
\column{0.50\linewidth}
\small
\begin{itemize}
\item World Wide Web Consortium
\item Tim Berners-Lee
\item Définit les standards du web
\item Procédures de gouvernance ouverte et prend ses décisions par consensus
\end{itemize}

\column{0.50\linewidth}
\includegraphics[width=0.80\textwidth]{03_Axe1_formats/Images/W3C.png}
\end{columns}
\vspace{5mm}
\textrightarrow{} \textit{(...) soumise à des enjeux économiques forts, notamment parce que les grands acteurs du web (Google, Facebook, Microsoft, etc.) peuvent imposer des standards de fait, sans passer par les procédures de normalisation internationale du W3C. (Dominique Cardon, Culture numérique.)}
    
\end{frame}

\begin{frame}{Les guerres de standards : le cas du livre numérique}

\begin{itemize}
    \item .epub : format standard, ouvert
    \item .azw : format lancé par Amazon, format propriétaire
\end{itemize}
\end{frame}

\begin{frame}{Apprendre à redévelopper une pensée critique de nos outils}
\begin{block}{}
    \centering
{\small \textit{La prolétarisation est, d’une manière générale, ce qui consiste à priver un sujet (producteur, consommateur, concepteur) de ses savoirs (savoir-faire, savoir-vivre, savoir concevoir et théoriser).}}\\
\textbf{Bernard Stiegler}
\end{block}
\begin{block}{}
\begin{itemize}
    \item Distinguer prolétarisation // paupérisation.
    \item Préférence pour \textbf{la maîtrise d'un outil} qui va dominer le marché plutôt que d'\textbf{enseigner les compétences} visées par ces outils (correction, édition, mise en forme etc...)\\
    \textrightarrow{} Éditeurs se trouvent "\textit{\textbf{prolétarisés}}"
\end{itemize}
\end{block}
\end{frame}

\begin{frame}{Expérience : révéler la composition d'un fichier Word}
\begin{itemize}
        \item Prendre un fichier Word (.docx) 
        \item Changer l'extension en.zip
        \item Ouvrir le fichier zip
    \end{itemize}
\begin{columns}[T,onlytextwidth]
\column{0.50\linewidth}
\centering
\includegraphics[width=5cm]{03_Axe1_formats/Images/experience1_bis.png}

\column{0.50\linewidth}
\centering
\includegraphics[width=6cm]{03_Axe1_formats/Images/experience1.png}
\end{columns}
    
\end{frame}

\begin{frame}{Expérience : révéler la composition d'un fichier Word}
\centering
\includegraphics[width=12cm]{03_Axe1_formats/Images/experience2.png}
\end{frame}

\begin{frame}{Expérience : révéler la composition d'un fichier Word}
\begin{block}{}
    \centering
\includegraphics[width=6cm]{03_Axe1_formats/Images/experience2_bis.png}
\includegraphics[width=6cm]{03_Axe1_formats/Images/experience3.png}
\end{block}
\vspace{5mm}
\textrightarrow{} Documents souvent mal structurés, difficilement interprétables par la machine sur le long terme, donc non interopérables et non pérennes.\\
\textrightarrow{} Éloignement progressif entre nos actions sur l'interface et la machine.

\end{frame}
\begin{frame}{Expérience : révéler la composition d'un fichier Word}

\begin{block}{Word:}
    \begin{itemize}
        \item Au début, développé avec son propre langage
        \item Création du .docx basé sur le langage XML
    \end{itemize}
\end{block}
\begin{block}{Formats ouverts:}
\begin{itemize}
     \item Fichiers encodés de façon transparente
    \item Code est dans le domaine public
    \item Interropérables
    \item A privilégier pour la préservation et le partage des données 
\end{itemize}
   
\end{block}

\begin{block}{En conclusion:}
    \begin{itemize}
        \item Docx = format propriétaire, payant, mais reste en théorie ouvert.
    \end{itemize}
\end{block}

\end{frame}

\section{Enjeux de pérennité des formats : que restera-t-il de la littérature numérique ?}
\begin{frame}{Des oeuvres éphémères?}

\begin{columns}[T,onlytextwidth]
\column{0.50\linewidth}
\centering
\includegraphics[width=.80\textwidth]{03_Axe1_formats/Images/bayard_livres_non_lus.png}

\column{0.50\linewidth}
\begin{itemize}
\item Pierre Bayard, 2007
    \item Essai qui prône la "non-lecture"
\end{itemize}
\end{columns}
\textbf{Raisons multiples} : oeuvres débranchées du web, oeuvres censurées. Mais très souvent des raisons d'obsolescence, liées au manque de pérennité des formats.
\end{frame}

\begin{frame}{Des oeuvres éphémères?}
    \begin{block}{Exemple: la fin de d'Adobe Flash Player}
        \begin{itemize}
            \item \textit{plug-in} permettant d'afficher/lire du contenu multimédia
            \item Début 2021: plus supporté par les navigateurs web
            \item Impossible de lire des images et des vidéos.
        \end{itemize}
    \end{block}
\textrightarrow{} En quelques semaines, la littérature numérique créée à l'aide de ce logiciel n'était plus accessible au public (Exemple \textit{Electronic Literature Organization}).
\end{frame}

\begin{frame}{Des oeuvres éphémères?}
\textbf{Création de WARC}
\begin{columns}[T,onlytextwidth]
\column{0.50\linewidth}
\begin{itemize}
    \item Publié à partir de 2009
    \item Développé dans le cadre des collectes de données du web
    \item Conçu pour la préservation du web
    \item A une notice internationale
\end{itemize}

\column{0.50\linewidth}
\centering
\includegraphics[width=.90\textwidth]{03_Axe1_formats/Images/WARC.png}
\end{columns}
\end{frame}

\begin{frame}{Les archives du web de la BNF}
    \textbf{Exemple de la collecte des Skyblogs}\\
    Emmanuelle Bermès, à la rencontre des premiers skyblogs archivés, 2025, \url{https://webcorpora.hypotheses.org/7348}\\
    \vspace{5mm}
    \centering
    \includegraphics[width=.80\textwidth]{03_Axe1_formats/Images/ex_skyblog.jpg}
\end{frame}

% Des oeuvres dont il nous reste des fragments (archives personnelles) Archives institutionnelles : BNF Format internationnal des archives = WARC. On est pas dans le paradigme de l'archive traditionnelle, qui est un prélèvement, mais dans une captation et représentation dans un autre format (un peu comme une photographie si l'on veut). Archives incomplètes, parfois fautives
% Des oeuvres que l'on essaie de restaurer, mais plus véritablement de re-créer, reconstruire dans d'autres formats. C'est ce qu'on appelle les émulateurs.

% De ce point de vue, un trauma dans la communauté elit = Flash

% Le 31 décembre 2020, Adobe a cessé de prendre en charge le logiciel Flash, une plateforme de premier plan pour l'art numérique très populaire à la fin du XXe siècle et au début du XXIe siècle. En quelques semaines, la littérature numérique créée à l'aide de ce logiciel n'était plus accessible au public, y compris les 447 œuvres que l'Electronic Literature Organization (ELO) avait rassemblées dans ses archives. À la fin du mois de janvier 2021, l'Electronic Literature Lab a commencé à travailler sérieusement à la restauration des archives Flash de l'ELO à l'aide de diverses méthodes : Ruffle.rs, Conifer, Webrecorder et des enregistrements vidéo réalisés avec le navigateur Pale Moon et la Wayback Machine.

% Réflexion sur notre dépendance à des formats qui ne sont pas standards, et qui appartiennent à des entreprises.

\end{document}