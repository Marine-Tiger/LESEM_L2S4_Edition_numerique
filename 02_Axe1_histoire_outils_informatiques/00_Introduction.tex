\documentclass[10pt]{beamer}

\documentclass[svgnames,smaller]{beamer}

\usetheme{metropolis}

\usepackage[utf8]{inputenc}
\usepackage[russian,french]{babel}

\usepackage[T1]{fontenc}
\usepackage{geometry}
\geometry{
    left=2.5mm,
    right=2.5mm,
    top=0mm,
    bottom=0mm
}
\usepackage{amssymb} % pour \blackdiamond
\usepackage{tikz}
\usepackage{graphicx,animate}

\newcommand{\replacementchar}{%
  \tikz[baseline=-0.6ex]{
    %\node[inner sep=0pt] (d) {\Large$\blackdiamond$};
    \node[inner sep=0pt] (d) {\Large$\blacksquare$};
    \node[white, font=\scriptsize\bfseries] at (d.center) {?};
  }%
}
\usepackage{tabularx}
%\usepackage[style=alphabetic,citestyle=authoryear]{biblatex}
%\addbibresource{bibliographie.bib}

\usepackage{xspace}
\newcommand{\themename}{\textbf{\textsc{metropolis}}\xspace}
\title{L2S4: Édition numérique\\
TD 2: Écrire avec des outils numériques\\
\small
Initiation à la notion de code informatique
}

\date{}
\author{Marine Tiger\\  \quad {marine.tiger@sorbonne-universite.fr\\}}

\institute{Littératures françaises et comparée ED19, CELLF, Sorbonne Université}

\begin{document}
\maketitle
\setbeamertemplate{section in toc}[sections numbered]

\section{Introduction}
\begin{frame}{Introduction}
    \begin{itemize}
        \item \textbf{Axe 1: Hardware, software, codes et formats : comprendre les enjeux des écritures numériques}
        \item Axe 2: Livrels, livres-appli, livres “augmentés”… comprendre les remédiations numériques du livre
        \item Axe 3: Éditorialisation littéraire et technologies du web
    \end{itemize}
\end{frame}

\begin{frame}{Introduction}
Quatre objectifs avec cette séance:
\begin{itemize}
    \item Comprendre ce qu'est le code informatique d'un point de vue technique;
    \item Comprendre en quoi cela fait évoluer notre conception de l'écriture à l'époque numérique:
    \item Discuter des enjeux liées à la compétence informatique par rapport aux compétences éditoriales et littéraires;
    \item Présenter certaines approches poétiques et théoriques.
\end{itemize}
\end{frame}

\section{Une littérature "électronique"}
\begin{frame}{Littérature "électronique"}
    \begin{itemize}
        \item Une écriture littéraire \textit{pré-web} qui questionne l'automatisation et le rapport à l'outil
        \item Premières œuvres: les \textbf{générateurs} et les \textbf{hypertextes}
        \item Supports externes (Disquettes, DVD, etc.)
    \end{itemize}

\end{frame}

\begin{frame}{Les générateurs automatiques de texte}

\begin{columns}[T,onlytextwidth]
\column{0.45\textwidth}
\begin{block}{}
\small
\begin{itemize}
        \item Travaillent le fantasme d'une écriture qui se produirait toute seule.
        \item Créent des textes à partir d'un dictionnaire de mots et d'une description informatique des règles d'assemblage de ces mots.\\
        \textrightarrow{} Totalement algorithmique.
    \end{itemize}
\end{block}

\column{0.55\textwidth}
\includegraphics[width=.86\textwidth]{02_Axe1_histoire_outils_informatiques/Images/magne.png}


\end{columns}
    
\end{frame}

\begin{frame}{Les hypertextes de fiction}
\centering
    \includegraphics[width=.45\textwidth]{02_Axe1_histoire_outils_informatiques/Images/Afternoon_cover.jpg}
    \includegraphics[width=.45\textwidth]{02_Axe1_histoire_outils_informatiques/Images/afternoon_love_story_choices.png}
    \textbf{Exemple}: \textit{Afternoon, a story} de Micheal Joyce.
\end{frame}

\begin{frame}{Une littérature indissociable de la technique}
\begin{block}{}
    \centering
    {\small
    \textit{À l'ère contemporaine, les textes imprimés et électroniques sont profondément imprégnés de code. Les technologies numériques sont désormais si étroitement intégrées aux processus d'impression commerciale que l'impression est davantage considérée comme une forme particulière de sortie de texte électronique que comme un support entièrement distinct. Néanmoins, le texte électronique reste distinct de l'impression dans la mesure où il n'est littéralement accessible qu'après avoir été exécuté par un code correctement exécuté.}}\\
    Katherine Hayles, \textit{« Electronic literature: What is it? »}, 2007, \url{https://eliterature.org/pad/elp.html}
\end{block}
\begin{block}{}
    \textrightarrow{} Définit la littérature électronique par sa capacité à être performée par la machine.
\end{block}

\end{frame}

\begin{frame}{Une littérature indissociable de la technique}
    \begin{block}{}
\centering
{\small
    \textit{Le lien direct entre le code et la performance du texte est fondamental pour comprendre la littérature électronique, en particulier pour apprécier sa spécificité en tant que production littéraire et technique. Les principaux genres du canon de la littérature électronique émergent non seulement des différentes façons dont l'utilisateur les expérimente, mais aussi de la structure et de la spécificité du code sous-jacent. Il n'est donc pas surprenant que certains genres soient désormais connus sous le nom du logiciel utilisé pour les créer et les exécuter.}}\\
    Katherine Hayles, \textit{« Electronic literature: What is it? »}, 2007, \url{https://eliterature.org/pad/elp.html}
\end{block}

\textrightarrow{} Établit un lien direct entre la compétence en écriture informatique et en écriture littéraire.
\end{frame}

\begin{frame}{\textit{"Si t'es pas codeur, t'es pas auteur..."}}

\begin{columns}[T,onlytextwidth]

\column{0.45\textwidth}
\begin{block}{}
\small
"\textit{Pour écrire un codex [numérique], l’auteur doit produire non seulement du texte et des liens, mais aussi du code informatique, un code qui fait partie intégrante de l’œuvre.}"
Thierry Crouzet, 2011, \url{https://tcrouzet.com/2011/03/04/pas-codeur-pas-auteur/}
\end{block}

\column{0.50\textwidth}
\includegraphics[width=.86\textwidth]{02_Axe1_histoire_outils_informatiques/Images/crouzet-auteur.png}
\end{columns}

\end{frame}

\begin{frame}{Problématiques}
\begin{itemize}
    \item Quelles compétences d'écriture de l'écrivain et de l'éditeur pour publier "numériquement"?
    \item Écrire du code pour être un écrivain/éditeur numérique?
    \item Rapport entre littérarité et littératie numérique? Peut-on évaluer un "style" de code?
\end{itemize}
    
\end{frame}

\section{Écritures [numériques]}
\begin{frame}{Écritures [numériques]}
\begin{itemize}
    \item Une technique d'inscription et d'enregistrement de la langue
    \item Une technique de maîtrise de la langue (le style)
"Bien écrire", "mal écrire"
\end{itemize}
    
\end{frame}

\begin{frame}{L'invention de l'écriture : retour sur une mutation épigénétique}

\begin{columns}[T,onlytextwidth]

\column{0.45\textwidth}
\begin{block}{}
    \begin{itemize}
    \item Une de nos premières technologies, à l'origine de toutes les autres
    \item \textbf{4000 - 3500 av JC} = écritures picto-idéographique
\item \textbf{3200 av JC} = hiéroglyphes
\item \textbf{1000 av JC} = écriture alphabétique (un code abstrait)
\end{itemize}
\end{block}


\column{0.50\textwidth}
\includegraphics[width=.86\textwidth]{02_Axe1_histoire_outils_informatiques/Images/hieroglyphes.jpeg}\vspace{1em}

\includegraphics[width=.86\textwidth]{02_Axe1_histoire_outils_informatiques/Images/alphabetGrec.png}
\end{columns}

\end{frame}



\section{L'écriture numérique : vers une mutation de la "raison graphique" et de la connaissance ?}
\begin{frame}{Jack Goody, \textit{La raison graphique: la domestication de la pensée sauvage}, 1977}

\begin{block}{}
    Approche anthropologique. Considère l'écriture comme une technique qui a suscité des mutations humaines profondes
\end{block}

\begin{block}{Se résume en trois points:}
    \begin{itemize}
        \item L'écriture est l'un des premiers moyens d'archivage des informations.
        \item L'écriture est donc aussi à l’origine d’un travail d’organisation du savoir en catégories. 
        \item L’écriture a permis le développement de la pensée logique, de l’abstraction et finalement de la science.
    \end{itemize}
\end{block}
\begin{block}{}
    \textbf{\textrightarrow{}} Impact des changements technologiques des techniques d'écriture à l'ère de la remédiation numérique sur nos modèles épistémologiques, nos connaissances et nos arts.
\end{block}
\end{frame}

\begin{frame}{Qu'est-ce que le code?}
\textbf{Exemple: \textit{Love Letters}, la première oeuvre de littérature numérique}

\begin{columns}[T,onlytextwidth]

\column{0.60\textwidth}
\small
\begin{itemize}
    \item Programme dont le code est écrit en langage python2 par Nick Montfort en 2018 (\textit{update} 2025), à partir d'une oeuvre de Christoper Strachey (1953, langage inconnu).
\end{itemize}
\includegraphics[width=0.85\textwidth]{02_Axe1_histoire_outils_informatiques/Images/love-letters.png}

\column{0.50\textwidth}
\includegraphics[width=0.75\textwidth]{02_Axe1_histoire_outils_informatiques/Images/love-letters-python.png}
\end{columns}
\end{frame}

\begin{frame}{Programme, code, langage ? Informatique?}
\begin{itemize}
    \item \textbf{Programme} = ensemble \textbf{de code exécutable}, rédigé dans un \textbf{langage de programmation} déterminé, avec son propre alphabet, son vocabulaire et ses règles de syntaxe.\\
    \item \textbf{Informatique} = Information + automatique. Calcul complexe dont l'exécution est délégué à une machine
    \item \textbf{Ordinateur} = machine à calculer
\end{itemize}
\end{frame}

\begin{frame}{Calcul? Machine?}
    \begin{block}{Calcul}
        \begin{itemize}
            \item quelle unité de mesure ?
            \item peut-on tout calculer ?
        \end{itemize}
    \end{block}
\begin{block}{Machine}
    \begin{itemize}
        \item quels outils ?
        \item comment mécaniser et automatiser ?
    \end{itemize}
\end{block}

\end{frame}

\begin{frame}{Le calcul...}
\begin{block}{}
\textbf{Code binaire}\\
    Système à deux symboles: O et 1. Combinés, ils permettent d'écrire des instructions compréhensibles pour l'ordinateur.\\

    \textbf{Exemple:}\\
    "Émile Zola" écrit en binaire = 01000101 01101101 01101001 01101100 01100101 00100000 01011010 01101111 01101100 01100001
\end{block}

\begin{block}{}
    \textrightarrow{} Permet d'écrire en langage naturel ou en langage informatique, des suites de symboles que l'ordinateur saura interpréter en 0/1, et ainsi saura manipuler (calculer), stocker (dans des disques, ou des cartes mémoires), et bien-sûr afficher en retour sur un écran pour les rendre lisibles à l'humain.
\end{block}

\end{frame}

\begin{frame}{... et la machine}
\begin{itemize}
    \item des constructions de l'esprit (histoire conceptuelle autant que technique)
    \item des réalisations industrielles (processus de mécanisation et d'automatisation)
\item des produits de consommation courante (à défaut d'une consommation "raisonnée")
\end{itemize}
    

\end{frame}

\begin{frame}{Atelier: Inspecter une page web pour en comprendre la structure}
    \begin{itemize}
        \item Ouvrez une page dans votre navigateur : par exemple, la page d'accueil de la faculté des lettres de SU.
        \item Depuis votre pavé tactile ou votre souris : clic gauche + inspecter (Raccourcis clavier sur la plupart des ordinateurs : ctrl+maj+c ou encore F12)
        \item Voici le résultat : vous êtes en face du code de la page.
        \item Dans la partie haute, vous allez retrouver le HTML (on reviendra sur ces noms de langage). Que contient-il ?

    \end{itemize}
\end{frame}

\begin{frame}{Atelier: Inspecter une page web pour en comprendre la structure}
\centering
    \includegraphics[width=12cm]{02_Axe1_histoire_outils_informatiques/Images/sorbonne_page.png}
\end{frame}

\begin{frame}{Atelier: Inspecter une page web pour en comprendre la structure}
\centering
    \includegraphics[width=12cm]{02_Axe1_histoire_outils_informatiques/Images/sorbonne_rose.png}
\end{frame}

\begin{frame}{Atelier: Inspecter une page web pour en comprendre la structure}
\centering
    \includegraphics[width=12cm]{02_Axe1_histoire_outils_informatiques/Images/sorbonne_image.png}
\end{frame}

\begin{frame}{Enjeux de l'écriture numérique}
    \begin{itemize}
        \item Enjeux d'écriture : maîtriser l'écrit
\item  Enjeux de lecture : accéder à l'ensemble des écrits
\item Enjeux de littératie : comprendre ce que l'on écrit
    \end{itemize}
\end{frame}

\section{Les écritures numériques : une réalité plurielle et multicouche}
\begin{frame}{Les écritures numériques: une réalité plurielle et multicouche}
    \begin{itemize}
        \item L'écriture numérique doit être déclinée au pluriel : elle est toujours le résultat d'une stratification
        \item Produire un document (textuel, visuel, sonore), revient à opérer un choix technique, derrière lequel se cache également un choix épistémologique, philosophique ou esthétique
    \end{itemize}
\end{frame}

\begin{frame}{Serge Bouchardon et Victor Petit, ou les trois "niveaux" de l'écriture numérique}
    \centering
    {\small \textit{Le livre a une \textbf{réalité physique ou matérielle} (le papier, l’encre) et une \textbf{réalité symbolique ou culturelle} (la langue, les signes à interpréter). Mais pour comprendre le fonctionnement du numérique il faut comprendre l’articulation, non pas de deux, mais de trois niveaux : il y a ce \textbf{qu’écrit la machine}, il y a ce \textbf{qu’écrit le programmeur de cette machine}, il y a ce \textbf{qu’écrit l’utilisateur de cette machine}. Lire un document numérique quelconque, c’est lire ces trois niveaux, quoique seul le dernier soit visible.}}\\
    \textbf{Bouchardon et Petit, 2017, \url{http://www.costech.utc.fr/CahiersCOSTECH/spip.php?article69}}
\end{frame}

\begin{frame}{Niveau 1: le code binaire, la technique comme matière}
\centering
{\small \textit{Le premier niveau de l’écriture numérique est d’abord théorique et repose sur la discrétisation et la manipulation d’unités formelles privées de sens (les 0 et les 1 ou n’importe quelles autres unités logiques formelles constituant un alphabet de manipulation). Tout contenu numérique peut être réduit en codage binaire, dont la signification éventuelle est arbitraire et indépendante de la manipulation formelle.}}\\
     \textbf{Bouchardon et Petit, 2017}
\end{frame}

\begin{frame}{Niveau 2: le code informatique, la technique comme code}
\centering
{\small \textit{Le deuxième niveau est celui du potentiel fonctionnel proposé par les applications, ce n’est plus le niveau de l’implémentation matérielle, mais le niveau du logiciel, celui de la manifestation, celui des formats d’écriture et des fonctions d’écriture. Comment nommer ce deuxième niveau de l’écriture numérique ? Nous proposons de l’appeler écriture pour les machines, soit l’écriture informatique ou l’écriture du code.}}\\
     \textbf{Bouchardon et Petit, 2017}
\end{frame}

\begin{frame}{Niveau 3: l'interface, la technique comme art}
\centering
{\small \textit{Ce troisième niveau est celui des utilisateurs du numérique, qui interprètent des formes sémiotiques et les manipulent, c’est le niveau de l’interaction (avec le niveau 1 via le niveau 2). Ce troisième niveau étant le plus usuel, nous pouvons le nommer écriture avec les machines. Mais tout l’enjeu de l’écriture numérique du troisième niveau est de ne pas oublier les deux autres niveaux, qui ne sont pas directement visibles mais qui rendent visible.}}\\
     \textbf{Bouchardon et Petit, 2017}
     
\end{frame}

\begin{frame}{Conclusion}
\centering
{\small \textit{Du fait de ces trois niveaux, il existe une tension propre à l’écriture numérique. S’il y a tension, c’est d’abord parce que l’écriture numérique réunit deux mondes jusqu’alors disjoints : le monde de l’écriture et le monde de la machine. [...] Pour ne parler ici que de la tension entre écriture et lecture, il est clair que la manipulation des signes et la dissémination des traces propres à l’écriture numérique entraînent ceci de particulier que non seulement écrire c’est lire, mais que lire c’est écrire. Avant d’être l’industrialisation de l’écriture, le web est l’industrialisation de la lecture.}}\\
     \textbf{Bouchardon et Petit, 2017}
\end{frame}

\begin{frame}{Lire-écrire // copier-coller : une écriture manipulable}
    \centering
{\small \textit{Plus que d’être du lisible, l’essence de l’écriture numérique est d’être du manipulable : tout peut devenir une unité de manipulation, le tout comme chaque partie ou ensemble de parties. La distinction entre partie et tout est d’ailleurs problématique dans le cas du numérique, car comme le rappelle Lev Manovich, les médias numériques possèdent une structure modulaire.}}\\
     \textbf{Bouchardon et Petit, 2017}
\end{frame}

\begin{frame}{Une écriture de plus en plus contrainte par les interfaces ?}
\centering
    \includegraphics[width=12cm]{02_Axe1_histoire_outils_informatiques/Images/wattpad.png}
\end{frame}

\begin{frame}{Un éloignement progressif du hardware au profit du software...}
    \centering
    {\small \textit{Aujourd'hui, ces raisons économiques impérieuses ont fait définitivement disparaître la modestie d'Alan Turing qui, à l'âge de pierre de l'histoire des ordinateurs, préférait lire les productions des machines en nombre binaires que décimaux. Au contraire, ce que l'on indûment la philosophie d'une communauté, elle-même appelée communauté informatique, met tout en oeuvre pour dissimuler le hardware derrière le logiciel, les signifiants électroniques derrière des interfaces homme-machine.}}\\
    \textbf{Friedrich Kittler, \textit{Mode protégé}, 1991 (2015).}
\end{frame}

\begin{frame}{... Et une marchandisation de l'écriture}
    \centering
    {\small \textit{Comme il est devenu possible d'importer des programmes d'un système à un autre, il est devenu économiquement attractif (au moins aux yeux de certains) de cacher le code de votre programme.}}\\
    \textbf{Lawrence Lessig, \textit{Culture libre}, 2004.}
\end{frame}

\begin{frame}{La bataille des éditeurs WYSIWYM vs WYSIWYG}
    \centering
    \includegraphics[width=6cm]{01-Introduction/Images/latex.png}
    \includegraphics[width=6cm]{01-Introduction/Images/page_garde.png}
\end{frame}


\section{Approches littéraires}
\begin{frame}{Les \textit{critical code studies}: lire le code, une compétence extra-informatique}
    \centering
    {\small \textit{L'exploration du code n'est jamais une activité neutre, libre d'une épistémologie ou d'une vision du monde : plutôt, elle se prête à la force interprétative des théories critiques, lesquelles peuvent être adoptées et adaptées pour produire des perspectives nouvelles et approfondies.}}\\
    \textbf{Mark Marino, \textit{critical code studies}, MIT Press, 2020}
\end{frame}

\begin{frame}{Une écriture sans écriture ? Kenneth Goldmisth}
    \begin{itemize}
        \item \textit{Uncreative Writing} (Columbia University press, 2011)
        \item Kenneth Goldsmith ubu-web - professeur de création littéraire
        \item \textit{L'écriture sans écriture}, traduction de François Bon aux éditions Jean Boîte (2018)
    \end{itemize}
\end{frame}

\begin{frame}{Les expériences d'écriture sans écriture de Kenneth Goldmisth}
    \begin{itemize}
        \item Choisir une image au hasard, en format .jpg
        \item Changer l'extension en .txt
        \item Ouvrir le fichier ainsi obtenu puis éditer à l'intérieur
        \item Rebasculer l'image dans son format original
        \item Tenter de l'ouvrir à nouveau
    \end{itemize}
\end{frame}

\begin{frame}{Les expériences d'écriture sans écriture de Kenneth Goldmisth}
    \centering
    \includegraphics[width=.80\textwidth]{02_Axe1_histoire_outils_informatiques/Images/clair_obscur.jpg}
    
\end{frame}

\begin{frame}{Les expériences d'écriture sans écriture de Kenneth Goldmisth}
    \centering
    \includegraphics[width=12cm]{02_Axe1_histoire_outils_informatiques/Images/code_img.png}
    
    \includegraphics[width=12cm]{02_Axe1_histoire_outils_informatiques/Images/code_img_modif.png}
    
\end{frame}

\begin{frame}{Les expériences d'écriture sans écriture de Kenneth Goldmisth}
    \centering
    \includegraphics[width=.80\textwidth]{02_Axe1_histoire_outils_informatiques/Images/after_modif.png} 
\end{frame}

\begin{frame}{Autorité et originalité ?}
\begin{block}{}
    \textbf{\textrightarrow{} L'édition et la curation comme nouvelle forme de création}
\end{block}
\centering
{\small \textit{How I make my way through this thicket of information —how I manage it, how I parse it, how I organize and distribute it— is what distinguishes my writing from yours}}\\
\textbf{Kenneth Goldsmith, \textit{L'écriture sans écriture}, Jean Boîte éditions}
\end{frame}

\end{document}