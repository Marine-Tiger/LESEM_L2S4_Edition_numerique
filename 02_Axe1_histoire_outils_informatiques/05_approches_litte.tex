\begin{frame}{Les \textit{critical code studies}: lire le code, une compétence extra-informatique}
    \centering
    {\small \textit{L'exploration du code n'est jamais une activité neutre, libre d'une épistémologie ou d'une vision du monde : plutôt, elle se prête à la force interprétative des théories critiques, lesquelles peuvent être adoptées et adaptées pour produire des perspectives nouvelles et approfondies.}}\\
    \textbf{Mark Marino, \textit{critical code studies}, MIT Press, 2020}
\end{frame}

\begin{frame}{Une écriture sans écriture ? Kenneth Goldmisth}
    \begin{itemize}
        \item \textit{Uncreative Writing} (Columbia University press, 2011)
        \item Kenneth Goldsmith ubu-web - professeur de création littéraire
        \item \textit{L'écriture sans écriture}, traduction de François Bon aux éditions Jean Boîte (2018)
    \end{itemize}
\end{frame}

\begin{frame}{Les expériences d'écriture sans écriture de Kenneth Goldmisth}
    \begin{itemize}
        \item Choisir une image au hasard, en format .jpg
        \item Changer l'extension en .txt
        \item Ouvrir le fichier ainsi obtenu puis éditer à l'intérieur
        \item Rebasculer l'image dans son format original
        \item Tenter de l'ouvrir à nouveau
    \end{itemize}
\end{frame}

\begin{frame}{Les expériences d'écriture sans écriture de Kenneth Goldmisth}
    \centering
    \includegraphics[width=.80\textwidth]{02_Axe1_histoire_outils_informatiques/Images/clair_obscur.jpg}
    
\end{frame}

\begin{frame}{Les expériences d'écriture sans écriture de Kenneth Goldmisth}
    \centering
    \includegraphics[width=12cm]{02_Axe1_histoire_outils_informatiques/Images/code_img.png}
    
    \includegraphics[width=12cm]{02_Axe1_histoire_outils_informatiques/Images/code_img_modif.png}
    
\end{frame}

\begin{frame}{Les expériences d'écriture sans écriture de Kenneth Goldmisth}
    \centering
    \includegraphics[width=.80\textwidth]{02_Axe1_histoire_outils_informatiques/Images/after_modif.png} 
\end{frame}

\begin{frame}{Autorité et originalité ?}
\begin{block}{}
    \textbf{\textrightarrow{} L'édition et la curation comme nouvelle forme de création}
\end{block}
\centering
{\small \textit{How I make my way through this thicket of information —how I manage it, how I parse it, how I organize and distribute it— is what distinguishes my writing from yours}}\\
\textbf{Kenneth Goldsmith, \textit{L'écriture sans écriture}, Jean Boîte éditions}
\end{frame}
