\begin{frame}{Jack Goody, \textit{La raison graphique: la domestication de la pensée sauvage}, 1977}

\begin{block}{}
    Approche anthropologique. Considère l'écriture comme une technique qui a suscité des mutations humaines profondes
\end{block}

\begin{block}{Se résume en trois points:}
    \begin{itemize}
        \item L'écriture est l'un des premiers moyens d'archivage des informations.
        \item L'écriture est donc aussi à l’origine d’un travail d’organisation du savoir en catégories. 
        \item L’écriture a permis le développement de la pensée logique, de l’abstraction et finalement de la science.
    \end{itemize}
\end{block}
\begin{block}{}
    \textbf{\textrightarrow{}} Impact des changements technologiques des techniques d'écriture à l'ère de la remédiation numérique sur nos modèles épistémologiques, nos connaissances et nos arts.
\end{block}
\end{frame}

\begin{frame}{Qu'est-ce que le code?}
\textbf{Exemple: \textit{Love Letters}, la première oeuvre de littérature numérique}

\begin{columns}[T,onlytextwidth]

\column{0.60\textwidth}
\small
\begin{itemize}
    \item Programme dont le code est écrit en langage python2 par Nick Montfort en 2018 (\textit{update} 2025), à partir d'une oeuvre de Christoper Strachey (1953, langage inconnu).
\end{itemize}
\includegraphics[width=0.85\textwidth]{02_Axe1_histoire_outils_informatiques/Images/love-letters.png}

\column{0.50\textwidth}
\includegraphics[width=0.75\textwidth]{02_Axe1_histoire_outils_informatiques/Images/love-letters-python.png}
\end{columns}
\end{frame}

\begin{frame}{Programme, code, langage ? Informatique?}
\begin{itemize}
    \item \textbf{Programme} = ensemble \textbf{de code exécutable}, rédigé dans un \textbf{langage de programmation} déterminé, avec son propre alphabet, son vocabulaire et ses règles de syntaxe.\\
    \item \textbf{Informatique} = Information + automatique. Calcul complexe dont l'exécution est délégué à une machine
    \item \textbf{Ordinateur} = machine à calculer
\end{itemize}
\end{frame}

\begin{frame}{Calcul? Machine?}
    \begin{block}{Calcul}
        \begin{itemize}
            \item quelle unité de mesure ?
            \item peut-on tout calculer ?
        \end{itemize}
    \end{block}
\begin{block}{Machine}
    \begin{itemize}
        \item quels outils ?
        \item comment mécaniser et automatiser ?
    \end{itemize}
\end{block}

\end{frame}

\begin{frame}{Le calcul...}
\begin{block}{}
\textbf{Code binaire}\\
    Système à deux symboles: O et 1. Combinés, ils permettent d'écrire des instructions compréhensibles pour l'ordinateur.\\

    \textbf{Exemple:}\\
    "Émile Zola" écrit en binaire = 01000101 01101101 01101001 01101100 01100101 00100000 01011010 01101111 01101100 01100001
\end{block}

\begin{block}{}
    \textrightarrow{} Permet d'écrire en langage naturel ou en langage informatique, des suites de symboles que l'ordinateur saura interpréter en 0/1, et ainsi saura manipuler (calculer), stocker (dans des disques, ou des cartes mémoires), et bien-sûr afficher en retour sur un écran pour les rendre lisibles à l'humain.
\end{block}

\end{frame}

\begin{frame}{... et la machine}
\begin{itemize}
    \item des constructions de l'esprit (histoire conceptuelle autant que technique)
    \item des réalisations industrielles (processus de mécanisation et d'automatisation)
\item des produits de consommation courante (à défaut d'une consommation "raisonnée")
\end{itemize}
    

\end{frame}

\begin{frame}{Atelier: Inspecter une page web pour en comprendre la structure}
    \begin{itemize}
        \item Ouvrez une page dans votre navigateur : par exemple, la page d'accueil de la faculté des lettres de SU.
        \item Depuis votre pavé tactile ou votre souris : clic gauche + inspecter (Raccourcis clavier sur la plupart des ordinateurs : ctrl+maj+c ou encore F12)
        \item Voici le résultat : vous êtes en face du code de la page.
        \item Dans la partie haute, vous allez retrouver le HTML (on reviendra sur ces noms de langage). Que contient-il ?

    \end{itemize}
\end{frame}

\begin{frame}{Atelier: Inspecter une page web pour en comprendre la structure}
\centering
    \includegraphics[width=12cm]{02_Axe1_histoire_outils_informatiques/Images/sorbonne_page.png}
\end{frame}

\begin{frame}{Atelier: Inspecter une page web pour en comprendre la structure}
\centering
    \includegraphics[width=12cm]{02_Axe1_histoire_outils_informatiques/Images/sorbonne_rose.png}
\end{frame}

\begin{frame}{Atelier: Inspecter une page web pour en comprendre la structure}
\centering
    \includegraphics[width=12cm]{02_Axe1_histoire_outils_informatiques/Images/sorbonne_image.png}
\end{frame}

\begin{frame}{Enjeux de l'écriture numérique}
    \begin{itemize}
        \item Enjeux d'écriture : maîtriser l'écrit
\item  Enjeux de lecture : accéder à l'ensemble des écrits
\item Enjeux de littératie : comprendre ce que l'on écrit
    \end{itemize}
\end{frame}