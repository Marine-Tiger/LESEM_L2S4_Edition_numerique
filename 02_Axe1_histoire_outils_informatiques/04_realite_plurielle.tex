\begin{frame}{Les écritures numériques: une réalité plurielle et multicouche}
    \begin{itemize}
        \item L'écriture numérique doit être déclinée au pluriel : elle est toujours le résultat d'une stratification
        \item Produire un document (textuel, visuel, sonore), revient à opérer un choix technique, derrière lequel se cache également un choix épistémologique, philosophique ou esthétique
    \end{itemize}
\end{frame}

\begin{frame}{Serge Bouchardon et Victor Petit, ou les trois "niveaux" de l'écriture numérique}
    \centering
    {\small \textit{Le livre a une \textbf{réalité physique ou matérielle} (le papier, l’encre) et une \textbf{réalité symbolique ou culturelle} (la langue, les signes à interpréter). Mais pour comprendre le fonctionnement du numérique il faut comprendre l’articulation, non pas de deux, mais de trois niveaux : il y a ce \textbf{qu’écrit la machine}, il y a ce \textbf{qu’écrit le programmeur de cette machine}, il y a ce \textbf{qu’écrit l’utilisateur de cette machine}. Lire un document numérique quelconque, c’est lire ces trois niveaux, quoique seul le dernier soit visible.}}\\
    \textbf{Bouchardon et Petit, 2017, \url{http://www.costech.utc.fr/CahiersCOSTECH/spip.php?article69}}
\end{frame}

\begin{frame}{Niveau 1: le code binaire, la technique comme matière}
\centering
{\small \textit{Le premier niveau de l’écriture numérique est d’abord théorique et repose sur la discrétisation et la manipulation d’unités formelles privées de sens (les 0 et les 1 ou n’importe quelles autres unités logiques formelles constituant un alphabet de manipulation). Tout contenu numérique peut être réduit en codage binaire, dont la signification éventuelle est arbitraire et indépendante de la manipulation formelle.}}\\
     \textbf{Bouchardon et Petit, 2017}
\end{frame}

\begin{frame}{Niveau 2: le code informatique, la technique comme code}
\centering
{\small \textit{Le deuxième niveau est celui du potentiel fonctionnel proposé par les applications, ce n’est plus le niveau de l’implémentation matérielle, mais le niveau du logiciel, celui de la manifestation, celui des formats d’écriture et des fonctions d’écriture. Comment nommer ce deuxième niveau de l’écriture numérique ? Nous proposons de l’appeler écriture pour les machines, soit l’écriture informatique ou l’écriture du code.}}\\
     \textbf{Bouchardon et Petit, 2017}
\end{frame}

\begin{frame}{Niveau 3: l'interface, la technique comme art}
\centering
{\small \textit{Ce troisième niveau est celui des utilisateurs du numérique, qui interprètent des formes sémiotiques et les manipulent, c’est le niveau de l’interaction (avec le niveau 1 via le niveau 2). Ce troisième niveau étant le plus usuel, nous pouvons le nommer écriture avec les machines. Mais tout l’enjeu de l’écriture numérique du troisième niveau est de ne pas oublier les deux autres niveaux, qui ne sont pas directement visibles mais qui rendent visible.}}\\
     \textbf{Bouchardon et Petit, 2017}
     
\end{frame}

\begin{frame}{Conclusion}
\centering
{\small \textit{Du fait de ces trois niveaux, il existe une tension propre à l’écriture numérique. S’il y a tension, c’est d’abord parce que l’écriture numérique réunit deux mondes jusqu’alors disjoints : le monde de l’écriture et le monde de la machine. [...] Pour ne parler ici que de la tension entre écriture et lecture, il est clair que la manipulation des signes et la dissémination des traces propres à l’écriture numérique entraînent ceci de particulier que non seulement écrire c’est lire, mais que lire c’est écrire. Avant d’être l’industrialisation de l’écriture, le web est l’industrialisation de la lecture.}}\\
     \textbf{Bouchardon et Petit, 2017}
\end{frame}

\begin{frame}{Lire-écrire // copier-coller : une écriture manipulable}
    \centering
{\small \textit{Plus que d’être du lisible, l’essence de l’écriture numérique est d’être du manipulable : tout peut devenir une unité de manipulation, le tout comme chaque partie ou ensemble de parties. La distinction entre partie et tout est d’ailleurs problématique dans le cas du numérique, car comme le rappelle Lev Manovich, les médias numériques possèdent une structure modulaire.}}\\
     \textbf{Bouchardon et Petit, 2017}
\end{frame}

\begin{frame}{Une écriture de plus en plus contrainte par les interfaces ?}
\centering
    \includegraphics[width=12cm]{02_Axe1_histoire_outils_informatiques/Images/wattpad.png}
\end{frame}

\begin{frame}{Un éloignement progressif du hardware au profit du software...}
    \centering
    {\small \textit{Aujourd'hui, ces raisons économiques impérieuses ont fait définitivement disparaître la modestie d'Alan Turing qui, à l'âge de pierre de l'histoire des ordinateurs, préférait lire les productions des machines en nombre binaires que décimaux. Au contraire, ce que l'on indûment la philosophie d'une communauté, elle-même appelée communauté informatique, met tout en oeuvre pour dissimuler le hardware derrière le logiciel, les signifiants électroniques derrière des interfaces homme-machine.}}\\
    \textbf{Friedrich Kittler, \textit{Mode protégé}, 1991 (2015).}
\end{frame}

\begin{frame}{... Et une marchandisation de l'écriture}
    \centering
    {\small \textit{Comme il est devenu possible d'importer des programmes d'un système à un autre, il est devenu économiquement attractif (au moins aux yeux de certains) de cacher le code de votre programme.}}\\
    \textbf{Lawrence Lessig, \textit{Culture libre}, 2004.}
\end{frame}

\begin{frame}{La bataille des éditeurs WYSIWYM vs WYSIWYG}
    \centering
    \includegraphics[width=6cm]{01-Introduction/Images/latex.png}
    \includegraphics[width=6cm]{01-Introduction/Images/page_garde.png}
\end{frame}
