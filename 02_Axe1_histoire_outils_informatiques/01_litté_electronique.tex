\begin{frame}{Littérature "électronique"}
    \begin{itemize}
        \item Une écriture littéraire \textit{pré-web} qui questionne l'automatisation et le rapport à l'outil
        \item Premières œuvres: les \textbf{générateurs} et les \textbf{hypertextes}
        \item Supports externes (Disquettes, DVD, etc.)
    \end{itemize}

\end{frame}

\begin{frame}{Les générateurs automatiques de texte}

\begin{columns}[T,onlytextwidth]
\column{0.45\textwidth}
\begin{block}{}
\small
\begin{itemize}
        \item Travaillent le fantasme d'une écriture qui se produirait toute seule.
        \item Créent des textes à partir d'un dictionnaire de mots et d'une description informatique des règles d'assemblage de ces mots.\\
        \textrightarrow{} Totalement algorithmique.
    \end{itemize}
\end{block}

\column{0.55\textwidth}
\includegraphics[width=.86\textwidth]{02_Axe1_histoire_outils_informatiques/Images/magne.png}


\end{columns}
    
\end{frame}

\begin{frame}{Les hypertextes de fiction}
\centering
    \includegraphics[width=.45\textwidth]{02_Axe1_histoire_outils_informatiques/Images/Afternoon_cover.jpg}
    \includegraphics[width=.45\textwidth]{02_Axe1_histoire_outils_informatiques/Images/afternoon_love_story_choices.png}
    \textbf{Exemple}: \textit{Afternoon, a story} de Micheal Joyce.
\end{frame}

\begin{frame}{Une littérature indissociable de la technique}
\begin{block}{}
    \centering
    {\small
    \textit{À l'ère contemporaine, les textes imprimés et électroniques sont profondément imprégnés de code. Les technologies numériques sont désormais si étroitement intégrées aux processus d'impression commerciale que l'impression est davantage considérée comme une forme particulière de sortie de texte électronique que comme un support entièrement distinct. Néanmoins, le texte électronique reste distinct de l'impression dans la mesure où il n'est littéralement accessible qu'après avoir été exécuté par un code correctement exécuté.}}\\
    Katherine Hayles, \textit{« Electronic literature: What is it? »}, 2007, \url{https://eliterature.org/pad/elp.html}
\end{block}
\begin{block}{}
    \textrightarrow{} Définit la littérature électronique par sa capacité à être performée par la machine.
\end{block}

\end{frame}

\begin{frame}{Une littérature indissociable de la technique}
    \begin{block}{}
\centering
{\small
    \textit{Le lien direct entre le code et la performance du texte est fondamental pour comprendre la littérature électronique, en particulier pour apprécier sa spécificité en tant que production littéraire et technique. Les principaux genres du canon de la littérature électronique émergent non seulement des différentes façons dont l'utilisateur les expérimente, mais aussi de la structure et de la spécificité du code sous-jacent. Il n'est donc pas surprenant que certains genres soient désormais connus sous le nom du logiciel utilisé pour les créer et les exécuter.}}\\
    Katherine Hayles, \textit{« Electronic literature: What is it? »}, 2007, \url{https://eliterature.org/pad/elp.html}
\end{block}

\textrightarrow{} Établit un lien direct entre la compétence en écriture informatique et en écriture littéraire.
\end{frame}

\begin{frame}{\textit{"Si t'es pas codeur, t'es pas auteur..."}}

\begin{columns}[T,onlytextwidth]

\column{0.45\textwidth}
\begin{block}{}
\small
"\textit{Pour écrire un codex [numérique], l’auteur doit produire non seulement du texte et des liens, mais aussi du code informatique, un code qui fait partie intégrante de l’œuvre.}"
Thierry Crouzet, 2011, \url{https://tcrouzet.com/2011/03/04/pas-codeur-pas-auteur/}
\end{block}

\column{0.50\textwidth}
\includegraphics[width=.86\textwidth]{02_Axe1_histoire_outils_informatiques/Images/crouzet-auteur.png}
\end{columns}

\end{frame}

\begin{frame}{Problématiques}
\begin{itemize}
    \item Quelles compétences d'écriture de l'écrivain et de l'éditeur pour publier "numériquement"?
    \item Écrire du code pour être un écrivain/éditeur numérique?
    \item Rapport entre littérarité et littératie numérique? Peut-on évaluer un "style" de code?
\end{itemize}
    
\end{frame}