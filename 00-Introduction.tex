\documentclass[10pt]{beamer}

\documentclass[svgnames,smaller]{beamer}

\usetheme{metropolis}

\usepackage[utf8]{inputenc}
\usepackage[russian,french]{babel}

\usepackage[T1]{fontenc}
\usepackage{geometry}
\geometry{
    left=2.5mm,
    right=2.5mm,
    top=0mm,
    bottom=0mm
}
\usepackage{amssymb} % pour \blackdiamond
\usepackage{tikz}
\usepackage{graphicx,animate}

\newcommand{\replacementchar}{%
  \tikz[baseline=-0.6ex]{
    %\node[inner sep=0pt] (d) {\Large$\blackdiamond$};
    \node[inner sep=0pt] (d) {\Large$\blacksquare$};
    \node[white, font=\scriptsize\bfseries] at (d.center) {?};
  }%
}
\usepackage{tabularx}
%\usepackage[style=alphabetic,citestyle=authoryear]{biblatex}
%\addbibresource{bibliographie.bib}

\usepackage{xspace}
\newcommand{\themename}{\textbf{\textsc{metropolis}}\xspace}
\title{L2S4: Édition numérique\\
TD 1: Introduction}

\date{}
\author{Marine Tiger\\  \quad {marine.tiger@sorbonne-universite.fr\\}}

\institute{Littératures françaises et comparée ED19, CELLF, Sorbonne Université}

\begin{document}
\maketitle
\setbeamertemplate{section in toc}[sections numbered]

\section{Introduction}
\begin{frame}{Introduction}
  
  % \tableofcontents[hideallsubsections]
  % \section{Introduction}
\begin{center}
    \textbf{Littérature? Édition? Numérique?}
\end{center}
\end{frame}

\begin{frame}{Introduction}
\begin{block}{Exercice: Dans ma "bibliothèque" de littérature numérique...}
\end{block}

Voici quelques objets présents dans la section "littérature numérique" de ma bibliothèque. Par groupes de 2, choisissez une référence et tentez de remplir le tableau descriptif.

\begin{center}
    \includegraphics[width=.35\textwidth]{01-Introduction/Images/biblio_num.png}
\end{center}

\end{frame}


\begin{frame}{Introduction}

\begin{block}{Des problématiques transversales:}
\end{block}

\begin{itemize}
    \item Quelles sont les compétences des auteurs et des éditeurs de ces objets ?
    \item Où est donc passé le livre : dans la tablette, le fichier, l'application, la plateforme ?
    \item Où est donc passé l'éditeur ? L'édition numérique est-elle une auto-édition ?
    \item De telles oeuvres relèvent-elles encore de la littérature, ou bien d'un autre média (audiovisuel, jeu vidéo, etc.) ?
    \item Quelle est la pérennité de ces objets numériques ?

\end{itemize}
\end{frame}

\begin{frame}{Introduction}

\begin{block}{Édition / Livre / Littérature : Numériques?}
\begin{itemize}
    \item \textbf{Édition numérique} : Ensemble des acteurs, outils, protocoles propre à l'énonciation éditoriale numérique.
    \item \textbf{Livre numérique} : désigne les différentes tentatives de remédiation du livre pour une lecture sur un support numérique.
    \item \textbf{Littérature numérique} : désigne les créations littéraires directement inscrites dans des environnements numériques.
\end{itemize}
\end{block}
\end{frame}

\section{L'édition numérique, "La révolution des révolutions ?"}
\begin{frame}{Une histoire des remédiations du livre}

\begin{center}
    \textit{"On présente le texte électronique comme une révolution. L'histoire du livre en a vu d'autres !"}\\
    Roger Chartier, Le livre en révolutions.
\end{center}

\end{frame}

\begin{frame}{Une histoire des remédiations du livre}


    \begin{itemize}
        \item Histoire matérielle et technique
\item Histoire culturelle de l'écriture et de la lecture
\item Histoire de la littérature
    \end{itemize}


\end{frame}

\begin{frame}{1ère époque: du \textit{volumen} au codex}


\begin{columns}[T,onlytextwidth]
\column{0.45\textwidth}
\includegraphics[width=.45\textwidth]{01-Introduction/Images/volumen2.jpeg}
\begin{block}{Volumen}
\small
\begin{itemize}
\item  rouleaux de plusieurs mètres que l'on déroule et renroule au fur et à mesure de manière horizontale (vers la droite)
\end{itemize}
\end{block}

\column{0.45\textwidth}
\includegraphics[width=.86\textwidth]{01-Introduction/Images/rouleau-codex.jpg}
\begin{block}{Codex}
\small
\begin{itemize}
\item mot latin, désigne le livre formé de feuilles pliées et assemblées en cahiers, et couvert d'une reliure tel que nous le connaissons
\end{itemize}
\end{block}
\end{columns}

\end{frame}


\begin{frame}{1ère époque: du \textit{volumen} au codex}

\begin{block}{Codex: Révolution technologique et intellectuelle}
    \begin{itemize}
        \item Invention de la page (et de l'architecture du savoir: sommaires, index, pages de titre et couverture)
        \item De nouveaux gestes de lecture et d'annotation
        \item La trace du volumen dans le lexique éditorial ("le volume")
    \end{itemize}
\end{block}

\end{frame}

\begin{frame}{2ème époque: du manuscrit à l'imprimé}

\begin{block}{Grâce à l'automatisation}
    \begin{itemize}
        \item Vers une stabilisation des textes
        \item Vers une uniformisation de l'édition
        \item Naissance d'une industrie culturelle
        \item Un changement d'échelle de la diffusion des savoirs
    \end{itemize}
\end{block}

\end{frame}

\begin{frame}{2ème époque: du manuscrit à l'imprimé}
\textbf{Les Incunables}\\
Les premiers livres fabriqués dans les décennies qui ont suivi l'invention de Gutenberg.
\begin{center}
    \includegraphics[width=.80\textwidth]{01-Introduction/Images/ex_incunable.png}
\end{center}    
\end{frame}

\begin{frame}{3ème époque: de l'imprimé au numérique (et vice-versa...)}
\begin{center}
    \includegraphics[width=.60\textwidth]{01-Introduction/Images/livrenum.jpeg}
\end{center}

\end{frame}

\begin{frame}{3ème époque: de l'imprimé au numérique (et vice-versa...)}
\begin{block}{Le "livre numérique" n'existe pas...}
    \begin{itemize}
        \item Une "adaptation" de l'imprimé à un mode de consultation/diffusion numérique (= livres homothétiques)
        \item Des propositions techniques et conceptuelles natives numériques (oeuvres hypermédiatiques, livres-appli, etc.)
        \item Une production hybride : complémentarité imprimé & numérique ; publication assistée par ordinateur (PAO); \textit{Web to print}
    \end{itemize}
\end{block}
\end{frame}

\begin{frame}{3ème époque: de l'imprimé au numérique (et vice-versa...)}
\textbf{De l'imprimé au numérique: ruptures et continuités}\\


\textit{L’inscription du texte sur l’écran crée une distribution, une organisation, une structuration du texte qui n’est pas du tout la même que celle que rencontrait le lecteur dans le rouleau de l’Antiquité ou le lecteur médiéval, moderne et contemporain dans le livre manuscrit ou imprimé, où le texte est organisé à partir de sa structure en cahiers, feuillets et pages.
la révolution du texte électronique est une révolution des structures du support matériel de l’écrit comme des manières de lire.}

Roger Chartier, Le livre en révolutions

\end{frame}

\begin{frame}{Le tournant numérique de la fabrique éditoriale : la PAO}
\textrightarrow{} Préparation des documents destinés à l'impression à l'aide d'un ordinateur en lieu et place des procédés historiques de la typographique et de la photocomposition.\\


\textbf{Ex:} Adobe In Design, Microsft Publisher, Word ...

\begin{center}
    \includegraphics[width=.35\textwidth]{01-Introduction/Images/prepaCopie_PAO.png}
    \includegraphics[width=.45\textwidth]{01-Introduction/Images/CouvIndesign-CreerCouv-1.jpg}
\end{center}
\end{frame}



\section{L'édition : une notion polysémique}
\begin{frame}{L'édition: une notion polysémique}

Le terme édition - tout comme celui d’éditeur - est bien plus polysémique qu’on ne l’imagine au premier abord. Tour d’horizon de ce que l’on entend par éditeur, et les activités d’édition.

    
\end{frame}

\begin{frame}{Établir, arranger et transmettre les textes : l'éditeur savant
}



\begin{columns}[T,onlytextwidth]
\column{0.57\textwidth}
\begin{block}
\small
\textit{« Personne qui fait paraître un texte après l’avoir établi »} (le philologue, le chercheur, etc.).
\end{block}
\begin{block}{Pourquoi a-t-on besoin d'un tel éditeur?}
    \begin{itemize}
        \item Garantir la transmission d'un texte ;
        \item Les textes ne sont pas, par nature, fixes. Ex: \href{https://www.dhi.ac.uk/onlinefroissart/browsey.jsp?f=b&pb3=Chi-1_128r&pb2=B88-1_13v&GlobalWord=46794&div0=ms.f.transc.Bes-1&panes=4&GlobalMode=shf&img3=&disp2=pb&disp3=pb&div3=ms.f.transc.Chi-1&div2=ms.f.transc.B88-1&div1=ms.f.transc.Aus&img0=&disp0=pb&disp1=pb&pb1=Aus_1_153v&img2=&GlobalShf=1-315&pb0=Bes-1_156v&img1=}{Les chroniques de Froissart}
    \end{itemize}
\end{block}

\column{0.50\textwidth}
\includegraphics[width=.60\textwidth]{01-Introduction/Images/Porphyry.jpg}
\end{columns}
\end{frame}

\begin{frame}{L'éditeur dans la chaîne éditoriale}

\begin{itemize}
    \item Éditeur= "Personne ou société qui assure la publication et la mise en vente d’un ouvrage imprimé", rouage essentiel de l'énonciation éditoriale
\end{itemize}

\begin{center}
    \includegraphics[width=.90\textwidth]{01-Introduction/Images/chaineMathay.png}
\end{center}
    
\end{frame}

\begin{frame}{L'éditeur dans la chaîne éditoriale}
    \begin{itemize}
        \item Éditeur transforme le texte (manuscrit) en ouvrage (livre). Ex: les romans-feuilletons.
        \item Éditeur est là pour transformer l'écrivain en auteur.
    \end{itemize}


\textrightarrow{} Un des rôles clefs de l'éditeur et de l'édition = \textbf{légitimer}.

\end{frame}



\section{L'édition: une histoire technique}
\begin{frame}{L'édition: une histoire technique}

\textrightarrow{} Un outil, un appareil technique pour produire et diffuser le texte.

\begin{block}{Outils d'édition possibles:}
\begin{itemize}
    \item Éditeur de texte ;
    \item Formateur de texte ;
    \item Traitement de texte.
\end{itemize}
\end{block}

\end{frame}

\begin{frame}{Éditeurs ou formateurs de texte}
\textrightarrow{} Outil qui met en forme un texte, à partir d'un fichier source, 
en texte brut mais contenant des indications de structures (balises).

\begin{center}
    \includegraphics[width=.50\textwidth]{01-Introduction/Images/latex.png}
    \includegraphics[width=.43\textwidth]{01-Introduction/Images/markdown.png}
\end{center}

\end{frame}

\begin{frame}{Traitement de texte}


\begin{itemize}
    \item Formatage du texte
    \item Affichage WYSIWYG (\textit{What You See Is What You Get}) du texte
    \item Impression finale
\end{itemize}
\begin{center}
    \includegraphics[width=.70\textwidth]{01-Introduction/Images/page_garde.png}
\end{center}


\textbf{\textrightarrow{} Imposent un format aux textes} 
\end{frame}


\section{Les trois fonctions éditoriales}
\begin{frame}{Les trois fonctions éditoriales}
\begin{center}
    \textit{L'édition peut être comprise comme un \textbf{processus de médiation} qui permet à un contenu d'exister et d'être accessible. On peut distinguer trois étapes de ce processus qui correspondent à trois fonctions différentes de l'édition : une fonction de \textbf{choix et production}, une fonction de \textbf{légitimation} et une fonction de \textbf{diffusion}. Analyser ces trois fonctions nous permettra de comprendre à quoi sert l'édition pour s'interroger ensuite sur la façon dont les technologies numériques réagencent le processus en le transformant.}\\
    Epron et Vitali-Rosati, L’édition à l’époque du numérique, 2017
\end{center}
\end{frame}

\begin{frame}{Production des contenus}
\begin{block}{}
    \textbf{Éditer}\\
    \small
Choisir et produire. Renvoie à la définition historique de l'éditeur (aujourd'hui éditeur scientifique ou philologue) = établissement du texte en vue de sa publication.\\
     Il ne s'agit donc pas seulement d'un travail sur le fond du texte, mais également d'une conceptualisation de sa forme livresque.
\end{block}


\begin{block}{}
\textbf{Production}\\
\small
    Ensemble des dispositifs dans leurs aspects humains, institutionnels et techniques qui concourent à la création des contenus livresques.
\end{block}
\end{frame}

\begin{frame}{Production des contenus}

\begin{block}{}
\textbf{Qu'est-ce qui est digne d'être éditer?}\\
\begin{itemize}
    \item Ce qui est intéressant?
    \item Ce qui va donc trouver un public?
\end{itemize}
\end{block}

\begin{block}{}
    La fonction éditoriale doit assurer la bonne mise en forme des contenus en vue de leur publication.
\end{block}
    
\end{frame}

\begin{frame}{Diffusion des contenus}
\begin{center}
    La fonction de diffusion comporte \textbf{l'adresse}, \textbf{la distribution} et tous les dispositifs qui tendent à rendre un contenu \textbf{matériellement accessible et visible}. Cette fonction comprend de fait l'ensemble des actions de \textbf{publicisation} du contenu. Il est important, pour un éditeur littéraire en particulier, d'apprendre à opérer la distinction entre le lecteur idéal supposé par toute oeuvre littéraire, une figure idéale / idéalisée notamment par l'auteur, de la communauté des lecteurs réels, avec leurs attentes et leurs usages culturels.
\end{center}
    
\end{frame}

\begin{frame}{Validation des contenus}

\begin{center}
    La fonction de légitimation renvoie aux dispositifs d'autorité qui permettent au public de \textbf{se repérer} dans les contenus, de \textbf{choisir les textes} et \textbf{d'évaluer leur fiabilité}. Si elle est supposée laisser des indices clairs sur leur valeur et finalement sur leur sens, on voit que la fonction de légitimation s'appuie sur un imaginaire collectif et individuel très fort.
\end{center}
    
\end{frame}

\section{Le "numérique": une notion polysémique}
\begin{frame}{Le "numérique" : une notion polysémique}
\begin{block}{}
    \textbf{Numérique (adjectif)}\\
Un outil numérique, une édition numérique, la culture numérique.
\end{block}    
    
\begin{block}{}
    \textbf{Numérique (nom)}\\
\textit{Le numérique est aujourd'hui dominé par l'essor des intelligences artificielles génératives (IA), qui envahissent les réseaux sociaux avec des vidéos et contenus générés par IA, comme en 2025 sur Instagram, TikTok et YouTube.} (Perplexity, via lemonde.fr)
\end{block}  

\end{frame}

\section{Objectifs du cours}
\begin{frame}{Objectifs du cours}
\begin{itemize}
    \item Découvrir la littérature numérique, notamment expérimentale
    \item Comprendre les grands enjeux de la transition numérique de l'édition
    \item Améliorer sa littératie numérique
\end{itemize}
  
\end{frame}

\end{document}